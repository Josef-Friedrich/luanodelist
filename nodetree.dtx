% \iffalse meta-comment
%
% Copyright (C) 2016 by Josef Friedrich <josef@friedrich.rocks>
% ----------------------------------------------------------------------
% This work may be distributed and/or modified under the conditions of
% the LaTeX Project Public License, either version 1.3 of this license
% or (at your option) any later version.  The latest version of this
% license is in:
%
%   http://www.latex-project.org/lppl.txt
%
% and version 1.3 or later is part of all distributions of LaTeX
% version 2005/12/01 or later.
%
% This work has the LPPL maintenance status `maintained'.
%
% The Current Maintainer of this work is Josef Friedrich.
%
% This work consists of the files nodetree.dtx and nodetree.ins
% and the derived filebase nodetree.sty and nodetree.lua.
%
% \fi
%
% \iffalse
%<*driver>
\ProvidesFile{nodetree.dtx}
%</driver>
%<package>\NeedsTeXFormat{LaTeX2e}[1999/12/01]
%<package>\ProvidesPackage{nodetree}
%<*package>
    [2016/07/13 Package to visualize node lists in a tree view]
%</package>
%<*driver>
\documentclass{ltxdoc}
\usepackage{paralist,fontspec,graphicx,fancyvrb}
\usepackage[
  colorlinks=true,
  linkcolor=red,
  filecolor=red,
  urlcolor=red,
]{hyperref}
\usepackage[documentationmode]{nodetree}
\EnableCrossrefs
\CodelineIndex
\RecordChanges

\setmonofont{DejaVu Sans Mono}

\def\nodetreelua#1{\texttt{\scantokens{\catcode`\_=12\relax#1}}}

\def\secref#1{(\rightarrow\ \ref{#1})}

\newcommand{\tmpgraphics}[1]{
  \noindent
  \includegraphics[scale=0.4]{graphics/#1}
}

\newcommand{\tmpexample}[1]{
\begin{nodetreeexample}
\input{#1_nodetree.tex}
\end{nodetreeexample}
}

\newcommand{\tmpverbexample}[1]{
\VerbatimInput[frame=single,fontsize=\footnotesize,firstline=4]{examples/#1.tex}
\tmpexample{#1}
}

\DefineVerbatimEnvironment{code}{Verbatim}
{
  frame=single,
  fontsize=\footnotesize,
}

\begin{document}

\providecommand*{\url}{\texttt}
\GetFileInfo{nodetree.dtx}
\title{The \textsf{nodetree} package}
\author{%
  Josef Friedrich\\%
  \url{josef@friedrich.rocks}\\%
  \href{https://github.com/Josef-Friedrich/nodetree}{github.com/Josef-Friedrich/nodetree}%
}
\date{\fileversion~from \filedate}

\maketitle

\tmpexample{packagename}

\newpage

\tableofcontents

\newpage

%-----------------------------------------------------------------------
% Abstract
%-----------------------------------------------------------------------

\section{Abstract}

|nodetree| is a development package that visualizes the structure of
node lists. |nodetree| shows its debug informations in the consoles’
output when you compile a Lua\TeX{} file. It uses a similar visual
representation for node lists as the UNIX |tree| command uses for a
folder structure.

Node lists are the main building blocks of each document generated by
the \TeX{} engine \emph{Lua\TeX}. The package |nodetree| doesn‘t change
the rendered document. The tree view can only be seen when using a
terminal to generate the document.

|nodetree| is inspired by a
\href{https://gist.github.com/pgundlach/556247}
{gist from Patrick Gundlach}.

%-----------------------------------------------------------------------
% Usage
%-----------------------------------------------------------------------

\section{Usage}

The package |nodetree| can be used both with Lua\TeX{} and Lua\LaTeX{}.
You have to use both engines in a text console. Run for example
|luatex luatex-test.tex| to list the nodes using Lua\TeX{}.

\begin{code}
%%
%% This is file `nodetree.tex',
%% generated with the docstrip utility.
%%
%% The original source files were:
%%
%% nodetree.dtx  (with options: `tex')
%% 
%% This is a generated file.
%% 
%% Copyright (C) 2015 by Josef Friedrich <josef@friedrich.rocks>
%% ----------------------------------------------------------------------
%% This work may be distributed and/or modified under the conditions of
%% the LaTeX Project Public License, either version 1.3c of this license
%% or (at your option) any later version. The latest version of this
%% license is in:
%% 
%%   http://www.latex-project.org/lppl.txt
%% 
%% and version 1.3c or later is part of all distributions of LaTeX
%% version 2008/05/05 or later.
%% 
\directlua{
  nodetree = require('nodetree')
  nodetree.set_option('engine', 'luatex')
}
\def\nodetreeoption[#1]#2{
  \directlua{
    nodetree.set_option('#1', '#2')
  }
}
\endinput
%%
%% End of file `nodetree.tex'.

\nodetreeregister{postline}

Lorem ipsum dolor.
\bye
\end{code}

Or run |lualatex lualatex-test.tex| to show a node tree using
Lua\LaTeX{}. In Lua\LaTeX{} you can omit |\nodetreeregister{postline}|.
|\usepackage{nodetree}| registers automatically the
|post_linebreak_filter|. If you don’t want debug the
|post_linebreak_filter| use |\nodetreeunregister{postline}|.

\begin{code}
\documentclass{article}
\usepackage{nodetree}

\begin{document}
Lorem ipsum dolor.
\end{document}
\end{code}

%%
% inside Lua code
%%

\subsection{Debug nodes inside Lua code}

Use the Lua function |nodetree.analyze(head)| to debug nodes inside your
Lua code. The following code snippet demonstrates the usage in Lua\TeX{}.
|head| is the current node.

\begin{code}
%%
%% This is file `nodetree.tex',
%% generated with the docstrip utility.
%%
%% The original source files were:
%%
%% nodetree.dtx  (with options: `tex')
%% 
%% This is a generated file.
%% 
%% Copyright (C) 2015 by Josef Friedrich <josef@friedrich.rocks>
%% ----------------------------------------------------------------------
%% This work may be distributed and/or modified under the conditions of
%% the LaTeX Project Public License, either version 1.3c of this license
%% or (at your option) any later version. The latest version of this
%% license is in:
%% 
%%   http://www.latex-project.org/lppl.txt
%% 
%% and version 1.3c or later is part of all distributions of LaTeX
%% version 2008/05/05 or later.
%% 
\directlua{
  nodetree = require('nodetree')
  nodetree.set_option('engine', 'luatex')
}
\def\nodetreeoption[#1]#2{
  \directlua{
    nodetree.set_option('#1', '#2')
  }
}
\endinput
%%
%% End of file `nodetree.tex'.


\directlua{
  local test = function (head)
    nodetree.analyze(head)
  end
  callback.register('post_linebreak_filter', test)
}

Lorem ipsum dolor.
\bye
\end{code}

This example illustrates how the function has to be applied in
Lua\LaTeX{}.

\begin{code}
\documentclass{article}
\usepackage{nodetree}

\begin{document}

\directlua{
  local test = function (head)
    nodetree.analyze(head)
  end
  luatexbase.add_to_callback('post_linebreak_filter', test, 'test')
}

Lorem ipsum dolor.
\end{document}
\end{code}

%-----------------------------------------------------------------------
% Macros
%-----------------------------------------------------------------------

\section{Macros}

%%
% \nodetreeregister
%%

\subsection{\cmd{\nodetreeregister}}

\DescribeMacro{\nodetreeregister}
\cmd{\nodetreeregister}\marg{callbacks}: The argument \marg{callbacks}
takes a comma separated list of callback aliases as described in
\secref{sec:option-callback}.

%%
% \nodetreeunregister
%%

\subsection{\cmd{\nodetreeunregister}}

\DescribeMacro{\nodetreeunregister}
\cmd{\nodetreeunregister}\marg{callbacks}: The argument \marg{callbacks}
takes a comma separated list of callback aliases as described in
\secref{sec:option-callback}.

%%
% \nodetreeoption
%%

\subsection{\cmd{\nodetreeoption}}

\DescribeMacro{\nodetreeoption}
\cmd{\nodetreeoption}\oarg{option}\marg{value}: \secref{sec:options}
This macro sets the option \oarg{option} to the value \marg{value}.

%%
% \nodetreeset
%%

\subsection{\cmd{\nodetreeset}}

\DescribeMacro{\nodetreeset}
\cmd{\nodetreeset}\marg{kv-options}:
This macro can only be used in Lua\LaTeX{}. \marg{kv-options} are key
value pairs.

\begin{code}
\nodetreeset{color=no,callbacks={hpack,vpack},verbosity=2}
\end{code}

%-----------------------------------------------------------------------
% Options
%-----------------------------------------------------------------------

\section{Options}
\label{sec:options}

%%
% callback
%%

\subsection{Option \texttt{callback}}
\label{sec:option-callback}

The option |callback| is the most important setting of the package. You
have to specify one alias to select the |callback|. Because of the
underscores the callback name contains it can not set by its technical
name (\rightarrow{} Figure \ref{fig:callback}).

This macros process callback options:
\cmd{\nodetreeregister}\marg{callbacks},
\cmd{\nodetreeunregister}\marg{callbacks},
\cmd{\nodetreeset}\marg{callback=<callbacks>} and
\cmd{\usepackage}\oarg{callback=<callbacks>}\marg{nodetree}.

Use commas to specify mulitple callbacks. Avoid using whitespaces:

\begin{code}
\nodetreeregister{preline,line,postline}
\end{code}

Wrap your callback aliases in curly braces for the macro |\nodetreeset|:

\begin{code}
\nodetreeset{callback={preline,line,postline}}
\end{code}

The same applies for the macro |\usepackage|:

\begin{code}
\usepackage{callback={preline,line,postline}}
\end{code}

%%
% Tabular callbacks
%%

\newcommand{\nodetreecallback}[3]{
  \nodetreelua{#1} & \nodetreelua{#2} & \nodetreelua{#3} \\
}

\begin{figure}

\noindent
\begin{tabular}{lll}
\textbf{Alias (short)} & \textbf{Alias (longer)} & \textbf{Callback} \\
\nodetreecallback{contribute}{contributefilter}{contribute_filter}
\nodetreecallback{buildpage}{buildpagefilter}{buildpage_filter}
\nodetreecallback{preline}{prelinebreakfilter}{pre_linebreak_filter}
\nodetreecallback{line}{linebreakfilter}{linebreak_filter}
\nodetreecallback{append}{appendtovlistfilter}{append_to_vlist_filter}
\nodetreecallback{postline}{postlinebreakfilter}{post_linebreak_filter}
\nodetreecallback{hpack}{hpackfilter}{hpack_filter}
\nodetreecallback{vpack}{vpackfilter}{vpack_filter}
\nodetreecallback{hpackq}{hpackquality}{hpack_quality}
\nodetreecallback{vpackq}{vpackquality}{vpack_quality}
\nodetreecallback{process}{processrule}{process_rule}
\nodetreecallback{preout}{preoutputfilter}{pre_output_filter}
\nodetreecallback{hyph}{hyphenate}{hyphenate}
\nodetreecallback{liga}{ligaturing}{ligaturing}
\nodetreecallback{kern}{kerning}{kerning}
\nodetreecallback{insert}{insertlocalpar}{insert_local_par}
\nodetreecallback{mhlist}{mlisttohlist}{mlist_to_hlist}
\end{tabular}

\caption{The callback aliases}
\label{fig:callback}
\end{figure}

%%
% channel
%%

\subsection{Option \texttt{channel}}
\label{sec:option-channel}

You can select the debug output channel with this option. The default
value for the option |channel| is |term| which displays the node tree in
the current terminal. Specify |log| and the package creates a log file
named |jobname_nodetree.log|. |jobname| is the name of your file you
want to debug.

%%
% verbosity
%

\subsection{Option \texttt{verbosity}}

Higher integer values result in a more verbose output. The default value
for this options is |1|. At the moment only verbosity level |2| is
implemented.

%%
% color
%%

\subsection{Option \texttt{color}}

The default option for |color| is |colored|. Use any other string (for
example |none| or |no|) to disable the colored terminal output of the
package.

\begin{code}
\usepackage[color=no]{nodetree}
\end{code}

%%
% unit
%%

\subsection{Option \texttt{unit}}

The option |unit| sets the length unit to display all length values of
the nodes. The default option for |unit| is |pt|. See figure
\ref{fig:fixed-units} and \ref{fig:relative-units} for possible values.

\begin{figure}
\begin{tabular}{lp{10cm}}
\textbf{Unit} &
\textbf{Description} \\

pt &
Point 1/72.27 inch. The conversion to metric units, to two decimal
places, is 1 point = 2.85 mm = 28.45 cm. \\

pc &
Pica, 12 pt \\

in &
Inch, 72.27 pt \\

bp &
Big point, 1/72 inch. This length is the definition of a point in
PostScript and many desktop publishing systems. \\

cm &
Centimeter \\

mm &
Millimeter \\

dd &
Didot point, 1.07 pt \\

cc &
Cicero, 12 dd \\

sp &
Scaled point, 1/65536 pt \\
\end{tabular}
\caption{Fixed units}
\label{fig:fixed-units}
\end{figure}

\begin{figure}
\begin{tabular}{lp{10cm}}
\textbf{Unit} &
\textbf{Description} \\

ex &
x-height of the current font \\

em &
Width of the capital letter M \\
\end{tabular}
\caption{Relative units}
\label{fig:relative-units}
\end{figure}

%%
% decimalplaces
%%

\subsection{Option \texttt{decimalplaces}}

The options |decimalplaces| sets the number of decimal places for some
node fields.

\begin{code}
\nodetreeoption[decimalplaces]{4}
\end{code}

gets

\begin{code}
├─GLYPH char: "a"; width: 5pt; height: 4.3055pt;
\end{code}

If |decimalplaces| is set to |0| only integer values are shown.

\begin{code}
├─GLYPH char: "a"; width: 5pt; height: 4pt;
\end{code}

%-----------------------------------------------------------------------
% Visual tree structure
%-----------------------------------------------------------------------

\section{Visual tree structure}

%%
% Two different connections
%%

\subsection{Two different connections}

Nodes in Lua\TeX{} are connected. The |nodetree| package distinguishs
between the |list| and |field| connections.

\begin{itemize}
 \item |list|: Nodes, which are double connected by |next| and
       |previous| fields.
 \item |field|: Connections to nodes by other fields than |next| and
       |previous| fields, e. g. |head|, |pre|.
\end{itemize}

%%
% Unicode characters
%%

\subsection{Unicode characters to show the tree view}

\renewcommand{\arraystretch}{1.5}

The package |nodetree| uses the unicode box drawing symbols. Your
default terminal font should contain this characters to obtain the tree
view. Eight box drawing characters are necessary.

\noindent
\begin{tabular}{lcl}
\textbf{Code} & \textbf{Character} & \textbf{Name} \\
U+2500 & |─| & BOX DRAWINGS LIGHT HORIZONTAL \\
U+2502 & |│| & BOX DRAWINGS LIGHT VERTICAL \\
U+2514 & |└| & BOX DRAWINGS LIGHT UP AND RIGHT \\
U+251C & |├| & BOX DRAWINGS LIGHT VERTICAL AND RIGHT \\
U+2550 & |═| & BOX DRAWINGS DOUBLE HORIZONTAL \\
U+2551 & |║| & BOX DRAWINGS DOUBLE VERTICAL \\
U+255A & |╚| & BOX DRAWINGS DOUBLE UP AND RIGHT \\
U+2560 & |╠| & BOX DRAWINGS DOUBLE VERTICAL AND RIGHT \\
\end{tabular}

For |list| connections \emph{light} characters are shown.

\begin{code}
│ │
│ ├─list1
│ └─list2
└─list3
\end{code}

|field| connections are visialized by \emph{Double} characters.

\begin{code}
║ ║
║ ╠═field1
║ ╚═field2
╚═field3
\end{code}

%-----------------------------------------------------------------------
% Examples
%-----------------------------------------------------------------------

\newpage

\section{Examples}

%%
% packagename
%%

\subsection{The node list of the package name}

\tmpverbexample{packagename}

%%
% math
%%

\newpage

\subsection{The node list of a mathematical formula}

\tmpverbexample{math}

%%
% ligatures
%%

\newpage

\subsection{The node list of the word \emph{Office}}

The characters \emph{ffi} are deeply nested in a discretionary node.

\tmpverbexample{ligatures}

%-----------------------------------------------------------------------
% Node types
%-----------------------------------------------------------------------

\subsection{Node types}

\newcommand{\tmpnodetypesub}[4]{
  \subsubsection{Type: #1(#2) Subtype: #3(#4)}
  \tmpverbexample{#2#1#4#3}
}

\newcommand{\tmpnodetype}[2]{
  \subsubsection{Type: #1(#2)}
  \tmpverbexample{#2#1}
}

\tmpnodetypesub{hlist}{0}{line}{1}
\tmpnodetypesub{hlist}{0}{box}{2}
\tmpnodetypesub{hlist}{0}{indent}{3}
\tmpnodetype{vlist}{1}
\tmpnodetype{rule}{2}
\tmpnodetype{mark}{4}
\tmpnodetypesub{disc}{7}{discretionary}{0}
\tmpnodetypesub{disc}{7}{regular}{3}
\tmpnodetypesub{whatsit}{8}{pdfaction}{22}
\tmpnodetypesub{whatsit}{8}{pdfcolorstack}{28}
\tmpnodetypesub{glue}{12}{baselineskip}{2}
\tmpnodetypesub{glue}{12}{parskip}{3}
\tmpnodetypesub{glue}{12}{spaceskip}{13}
\tmpnodetypesub{glue}{12}{leaders}{100}
\tmpnodetypesub{glue}{12}{cleaders}{101}
\tmpnodetypesub{glue}{12}{xleaders}{102}
\tmpnodetypesub{kern}{13}{userkern}{0}
\tmpnodetypesub{kern}{13}{fontkern}{1}
\tmpnodetypesub{kern}{13}{accentkern}{2}
\tmpnodetypesub{kern}{13}{italiccorrection}{3}
\tmpnodetype{penalty}{14}
\tmpnodetype{glyph}{29}
\tmpnodetype{attribute}{38}
\tmpnodetype{attributelist}{40}

%-----------------------------------------------------------------------
% Index
%-----------------------------------------------------------------------

  \DocInput{nodetree.dtx}
  \pagebreak
  \PrintChanges
  \pagebreak
  \PrintIndex
\end{document}
%</driver>
%<*readme>

![nodetree](graphics/packagename.png)

# Abstract

`nodetree` is a development package that visualizes the structure of
node lists. `nodetree` shows its debug informations in the consoles’
output when you compile a LuaTeX file. It uses a similar visual
representation for node lists as the UNIX `tree` command uses for a
folder structure.

Node lists are the main building blocks of each document generated by
the TeX engine LuaTeX. The package `nodetree` doesn‘t change
the rendered document. The tree view can only be seen when using a
terminal to generate the document.

`nodetree` is inspired by a
[gist from Patrick Gundlach](https://gist.github.com/pgundlach/556247).

# License

Copyright (C) 2016 by Josef Friedrich <josef@friedrich.rocks>
------------------------------------------------------------------------
This work may be distributed and/or modified under the conditions of
the LaTeX Project Public License, either version 1.3 of this license
or (at your option) any later version.  The latest version of this
license is in:

  http://www.latex-project.org/lppl.txt

and version 1.3 or later is part of all distributions of LaTeX
version 2005/12/01 or later.

# CTAN

Since July 2016 the cloze package is included in the Comprehensive TeX
Archive Network (CTAN).

* TeX archive: http://mirror.ctan.org/tex-archive/macros/luatex/generic/nodetree
* Package page: http://www.ctan.org/pkg/nodetree

# Repository

https://github.com/Josef-Friedrich/nodetree

# Installation

Get source:

    git clone git@github.com:Josef-Friedrich/nodetree.git
    cd nodetree

Compile:

    make

or manually:

    luatex nodetree.ins
    lualatex nodetree.dtx
    makeindex -s gglo.ist -o nodetree.gls nodetree.glo
    makeindex -s gind.ist -o nodetree.ind nodetree.idx
    lualatex nodetree.dtx

# Examples

## The node list of the package name

```latex
\documentclass{article}
\usepackage{nodetree}
\begin{document}
nodetree
\end{document}
```

![nodetree](graphics/packagename.png)

## The node list of a mathematical formula

```latex
\documentclass{article}
\usepackage[callback={mhlist}]{nodetree}
\begin{document}
\[\left(a\right)\left[\frac{b}{a}\right]=a\,\]
\end{document}
```

![nodetree](graphics/math.png)

## The node list of the word 'Office'

The characters 'ffi' are deeply nested in a discretionary node.

```latex
\documentclass{article}
\usepackage{nodetree}
\begin{document}
Office
\end{document}
```

![nodetree](graphics/ligatures.png)

%</readme>
% \fi
%
% \CheckSum{0}
%
% \CharacterTable
%  {Upper-case    \A\B\C\D\E\F\G\H\I\J\K\L\M\N\O\P\Q\R\S\T\U\V\W\X\Y\Z
%   Lower-case    \a\b\c\d\e\f\g\h\i\j\k\l\m\n\o\p\q\r\s\t\u\v\w\x\y\z
%   Digits        \0\1\2\3\4\5\6\7\8\9
%   Exclamation   \!     Double quote  \"     Hash (number) \#
%   Dollar        \$     Percent       \%     Ampersand     \&
%   Acute accent  \'     Left paren    \(     Right paren   \)
%   Asterisk      \*     Plus          \+     Comma         \,
%   Minus         \-     Point         \.     Solidus       \/
%   Colon         \:     Semicolon     \;     Less than     \<
%   Equals        \=     Greater than  \>     Question mark \?
%   Commercial at \@     Left bracket  \[     Backslash     \\
%   Right bracket \]     Circumflex    \^     Underscore    \_
%   Grave accent  \`     Left brace    \{     Vertical bar  \|
%   Right brace   \}     Tilde         \~}
%
%
% \changes{v0.1}{2015/06/16}{Converted to DTX file}
% \changes{v1.0}{2016/07/07}{Inital release}
% \changes{v1.1}{2016/07/13}{Fix the registration of same callbacks}
%
% \DoNotIndex{\newcommand,\newenvironment,\def,\directlua}
%
% \StopEventually{}
% \pagebreak
% \section{Implementation}
%
% \iffalse
%<*tex>
% \fi
% \MacroTopsep = 10pt plus 2pt minus 2pt
% \MacrocodeTopsep = 10pt plus 1.2pt minus 1pt
% \makeatletter
% \c@CodelineNo 25 \relax
% \makeatother
%
% \subsection{The file \tt{nodetree.tex}}
%
%    \begin{macrocode}
\directlua{
  nodetree = require('nodetree')
  nodetree.set_option('engine', 'luatex')
  nodetree.set_default_options()
}
%    \end{macrocode}
%
% \begin{macro}{\nodetreeoption}
%    \begin{macrocode}
\def\nodetreeoption[#1]#2{
  \directlua{
    nodetree.set_option('#1', '#2')
  }
}
%    \end{macrocode}
% \end{macro}
%
% \begin{macro}{\nodetreeregister}
%    \begin{macrocode}
\def\nodetreeregister#1{
  \directlua{
    nodetree.set_option('callback', '#1')
    nodetree.register_callbacks()
  }
}
%    \end{macrocode}
% \end{macro}
%
% \begin{macro}{\nodetreeunregister}
%    \begin{macrocode}
\def\nodetreeunregister#1{
  \directlua{
    nodetree.set_option('callback', '#1')
    nodetree.unregister_callbacks()
  }
}
%    \end{macrocode}
% \end{macro}
%
% \iffalse
%</tex>
%<*package>
% \fi
% \makeatletter
% \c@CodelineNo 25 \relax
% \makeatother
%
% \subsection{The file \tt{nodetree.sty}}
%
%    \begin{macrocode}
% \iffalse meta-comment
%
% Copyright (C) 2015 by Josef Friedrich <josef@friedrich.rocks>
% ----------------------------------------------------------------------
% This work may be distributed and/or modified under the conditions of
% the LaTeX Project Public License, either version 1.3 of this license
% or (at your option) any later version.  The latest version of this
% license is in:
%
%   http://www.latex-project.org/lppl.txt
%
% and version 1.3 or later is part of all distributions of LaTeX
% version 2005/12/01 or later.
%
% This work has the LPPL maintenance status `maintained'.
%
% The Current Maintainer of this work is Josef Friedrich.
%
% This work consists of the files nodetree.dtx and nodetree.ins
% and the derived filebase nodetree.sty and nodetree.lua.
%
% \fi
%
% \iffalse
%<*driver>
\ProvidesFile{nodetree.dtx}
%</driver>
%<package>\NeedsTeXFormat{LaTeX2e}[1999/12/01]
%<package>\ProvidesPackage{nodetree}
%<*package>
    [2015/11/13 Package to debug node lists used by LuaTeX]
%</package>
%<*driver>
\documentclass{ltxdoc}
\usepackage{hyperref,paralist}
\EnableCrossrefs
\CodelineIndex
\RecordChanges
\begin{document}

\providecommand*{\url}{\texttt}
\GetFileInfo{nodetree.dtx}
\title{The \textsf{nodetree} package}
\author{%
  Josef Friedrich\\%
  \url{josef@friedrich.rocks}\\%
  \href{https://github.com/Josef-Friedrich/cloze}{github.com/Josef-Friedrich/nodetree}%
}
\date{\fileversion~from \filedate}

\maketitle

\tableofcontents

\section{Option 'channel'}

Value "term"

\begin{verbatim}
\usepackage[channel=term]{nodetree}
\end{verbatim}

Value "log"

\begin{verbatim}
\usepackage[channel=log]{nodetree}
\end{verbatim}

Value "term and log"

\begin{verbatim}
\usepackage[channel={term and log}]{nodetree}
\end{verbatim}

  \DocInput{nodetree.dtx}
  \pagebreak
  \PrintChanges
  \pagebreak
  \PrintIndex
\end{document}
%</driver>
%<*readme>
# nodetree

Inspired by a [gist of Patrick Gundlach](https://gist.github.com/pgundlach/556247).

`nodetree` displays some debug informations of the node list in the
terminal, when you render a Latex document.

```
post_linebreak_filter:
│
├─GLUE subtype: baselineskip; width: 5.06pt;
└─HLIST subtype: line; width: 345pt; height: 6.94pt; dir: TLT; glue_order: 2; glue_sign: 1; glue_set: 304.99993896484;
 ╚═head:
  ├─LOCAL_PAR dir: TLT;
  ├─HLIST subtype: indent; width: 15pt; dir: TLT;
  ├─GLYPH char: "O"; font: 15; left: 2; right: 3; uchyph: 1; width: 7.78pt; height: 6.83pt;
  ├─DISC subtype: regular; penalty: 50;
  │ ╠═post:
  │ ║ └─GLYPH subtype: ghost; char: "\12"; font: 15; width: 5.56pt; height: 6.94pt;
  │ ║  ╚═components:
  │ ║   ├─GLYPH subtype: ligature; char: "f"; font: 15; left: 2; right: 3; uchyph: 1; width: 3.06pt; height: 6.94pt;
  │ ║   └─GLYPH subtype: ligature; char: "i"; font: 15; left: 2; right: 3; uchyph: 1; width: 2.78pt; height: 6.68pt;
  │ ╠═pre:
  │ ║ ├─GLYPH char: "f"; font: 15; left: 2; right: 3; uchyph: 1; width: 3.06pt; height: 6.94pt;
  │ ║ └─GLYPH char: "-"; font: 15; left: 2; right: 3; uchyph: 1; width: 3.33pt; height: 4.31pt;
  │ ╚═replace:
  │  └─GLYPH subtype: ghost; char: "\14"; font: 15; width: 8.33pt; height: 6.94pt;
  │   ╚═components:
  │    ├─GLYPH subtype: ghost; char: "\11"; font: 15; width: 5.83pt; height: 6.94pt;
  │    │ ╚═components:
  │    │  ├─GLYPH subtype: ligature; char: "f"; font: 15; left: 2; right: 3; uchyph: 1; width: 3.06pt; height: 6.94pt;
  │    │  └─GLYPH subtype: ligature; char: "f"; font: 15; left: 2; right: 3; uchyph: 1; width: 3.06pt; height: 6.94pt;
  │    └─GLYPH subtype: ligature; char: "i"; font: 15; left: 2; right: 3; uchyph: 1; width: 2.78pt; height: 6.68pt;
  ├─GLYPH char: "c"; font: 15; left: 2; right: 3; uchyph: 1; width: 4.44pt; height: 4.31pt;
  ├─GLYPH char: "e"; font: 15; left: 2; right: 3; uchyph: 1; width: 4.44pt; height: 4.31pt;
  ├─PENALTY penalty: 10000;
  ├─GLUE subtype: parfillskip; stretch: 65536; stretch_order: 2;
  └─GLUE subtype: rightskip;

```

# UTF8 Box drawing symbols

## Light

```
│ │
│ ├─┤field1: 1pt├┤field2: 1pt│
│ └─
└─
```

## Heavy

```
┃ ┃
┃ ┣━┫field1: 1pt┣┫field2: 1pt┃
┃ ┗━
┗━
```

## Double

```
║ ║
║ ╠═╣field1: 1pt╠╣field2: 1pt║
║ ╚═
╚═
```

%</readme>
% \fi
%
% \CheckSum{0}
%
% \CharacterTable
%  {Upper-case    \A\B\C\D\E\F\G\H\I\J\K\L\M\N\O\P\Q\R\S\T\U\V\W\X\Y\Z
%   Lower-case    \a\b\c\d\e\f\g\h\i\j\k\l\m\n\o\p\q\r\s\t\u\v\w\x\y\z
%   Digits        \0\1\2\3\4\5\6\7\8\9
%   Exclamation   \!     Double quote  \"     Hash (number) \#
%   Dollar        \$     Percent       \%     Ampersand     \&
%   Acute accent  \'     Left paren    \(     Right paren   \)
%   Asterisk      \*     Plus          \+     Comma         \,
%   Minus         \-     Point         \.     Solidus       \/
%   Colon         \:     Semicolon     \;     Less than     \<
%   Equals        \=     Greater than  \>     Question mark \?
%   Commercial at \@     Left bracket  \[     Backslash     \\
%   Right bracket \]     Circumflex    \^     Underscore    \_
%   Grave accent  \`     Left brace    \{     Vertical bar  \|
%   Right brace   \}     Tilde         \~}
%
%
% \changes{v0.1}{2015/06/16}{Converted to DTX file}
% \changes{v1.0}{2015/07/08}{Inital release}
%
% \DoNotIndex{\newcommand,\newenvironment,\def,\directlua}
%
% \StopEventually{}
% \pagebreak
% \section{Implementation}
%
% \iffalse
%<*tex>
% \fi
% \MacroTopsep = 10pt plus 2pt minus 2pt
% \MacrocodeTopsep = 10pt plus 1.2pt minus 1pt
% \makeatletter
% \c@CodelineNo 25 \relax
% \makeatother
%
% \subsection{The file \tt{nodetree.tex}}
%
%    \begin{macrocode}
\directlua{
  nodetree = require('nodetree')
}
\def\nodetreeoption[#1]#2{
  \directlua{
    nodetree.set_option('#1', '#2')
  }
}
\def\nodetreeshowoption#1{
  \directlua{
    tex.print(nodetree.get_option('#1'))
  }
}
\def\nodetreeprocessoptions{
  \nodetreeoption[engine]{luatex}
  \directlua{
    nodetree.set_default_options()
    nodetree.register_callbacks()
  }
}
\def\nodetreeregister{%
  \strut%
  \directlua{nodetree.marker('start')}%
}
\def\nodetreeunregister{%
  \directlua{nodetree.marker('stop')}%
  \strut%
}
%    \end{macrocode}
%    \end{macrocode}
%
% \iffalse
%</tex>
%<*package>
% \fi
% \makeatletter
% \c@CodelineNo 25 \relax
% \makeatother
%
% \subsection{The file \tt{nodetree.sty}}
%
%    \begin{macrocode}
\directlua{
  nodetree = require('nodetree')
  nodetree.set_option('engine', 'lualatex')
}
%    \end{macrocode}
%    \begin{macrocode}
\RequirePackage{kvoptions}
%    \end{macrocode}
%
%    \begin{macrocode}
\SetupKeyvalOptions{
  family=NT,
  prefix=NT@
}
%    \end{macrocode}
%
%    \begin{macrocode}
\def\NT@set@option[#1]#2{%
  \directlua{nodetree.set_option('#1', '#2')}%
}
\DeclareStringOption[colored]{color}
\define@key{NT}{color}[]{\NT@set@option[color]{#1}}
%    \end{macrocode}
%    \begin{macrocode}
\DeclareStringOption[term and log]{channel}
\define@key{NT}{channel}[]{\NT@set@option[channel]{#1}}
%    \end{macrocode}
%
%    \begin{macrocode}
\DeclareStringOption[postlinebreak]{callback}
\define@key{NT}{callback}[]{\NT@set@option[callback]{#1}}
%    \end{macrocode}
%
%    \begin{macrocode}
\DeclareStringOption[1]{verbosity}
\define@key{NT}{verbosity}[]{\NT@set@option[verbosity]{#1}}
%    \end{macrocode}
%
%    \begin{macrocode}
\DeclareVoidOption{global}{
  \NT@set@option[global]{true}
  \NT@set@option[print]{start}
}
%    \end{macrocode}
%
%    \begin{macrocode}
\ProcessKeyvalOptions*
\directlua{
  nodetree.set_default_options()
  nodetree.register_callbacks()
}
%    \end{macrocode}
%
%    \begin{macrocode}
\newcommand{\nodetreeset}[1]{\setkeys{nodetree}{#1}}
%    \end{macrocode}
%
% \begin{environment}{nodetreeenv}
%    \begin{macrocode}
\newenvironment{nodetreeenv}{%
  \strut%
  \directlua{
    nodetree.marker('start')
  }%
}{%
  \directlua{
    nodetree.marker('stop')
  }%
  \strut%
}
%    \end{macrocode}
% \end{environment}
% \iffalse
%</package>
%<*luamain>
% \fi
%
% \makeatletter
% \c@CodelineNo 0 \relax
% \makeatother
%
% \subsection{The file \tt{nodetree.lua}}
%
%
%    \begin{macrocode}
local nodex = {}
%    \end{macrocode}
%
%    \begin{macrocode}
local template = {}
%    \end{macrocode}
%
%    \begin{macrocode}
local nodetree = {}
%    \end{macrocode}
%
% Nodes in Lua\TeX are connected. The nodetree view distinguishs betweens
% the |list| and |field| connections.

% \begin{itemize}
%  \item |list|: Nodes, which are double connected by |next| and
%        |previous| fields.
%  \item |field|: Connections to nodes by other fields than |next| and
%        |previous| fields, e. g. |head|, |pre|.
% \end{itemize}
%
% The lua table named |nodetree| holds states values for the present nodetree
% item.
% \begin{verbatim}
%  nodetree:
%    - 1:
%      - list: continue
%      - field: stop
%    - 2:
%      - list: continue
%      - field: stop
% \end{verbatim}
%    \begin{macrocode}
nodetree.state = {}
%    \end{macrocode}
%
%    \begin{macrocode}
local base = {}
%    \end{macrocode}
%
%    \begin{macrocode}
local options = {}
%    \end{macrocode}
%
% \subsubsection{nodex --- node extended}
%
% Get the node id form, e. g.:
% \begin{verbatim}
% <node    nil <    172 >    nil : hlist 2>
% \end{verbatim}
%    \begin{macrocode}
function nodex.node_id(n)
  return string.gsub(tostring(n), '^<node%s+%S+%s+<%s+(%d+).*', '%1')
end
%    \end{macrocode}
%
%    \begin{macrocode}
function nodex.create_marker(string)
  local marker = node.new('whatsit','user_defined')
  marker.type = 115
  marker.user_id = options.user_id
  marker.value = string
  node.write(marker)
end
%    \end{macrocode}
%
%    \begin{macrocode}
function nodex.subtype(n)
  local typ = node.type(n.id)

  local subtypes = {
%    \end{macrocode}
% \paragraph{hlist (0)}
%    \begin{macrocode}
    hlist = {
      [0] = 'unknown',
      [1] = 'line',
      [2] = 'box',
      [3] = 'indent',
      [4] = 'alignment',
      [5] = 'cell',
      [6] = 'equation',
      [7] = 'equationnumber',
    },
%    \end{macrocode}
% \paragraph{vlist (1)}
%    \begin{macrocode}
    vlist = {
      [0] = 'unknown',
      [4] = 'alignment',
      [5] = 'cell',
    },
%    \end{macrocode}
% \paragraph{rule (2)}
%    \begin{macrocode}
    rule = {
      [0] = 'unknown',
      [1] = 'box',
      [2] = 'image',
      [3] = 'empty',
      [4] = 'user',
    },
%    \end{macrocode}
%
% \noindent
% Nodes without subtypes:
% \begin{compactitem}
% \item ins (3)
% \item mark (4)
% \end{compactitem}
%    \begin{macrocode}
%    \end{macrocode}
% \paragraph{adjust (5)}
%    \begin{macrocode}
    adjust = {
      [0] = 'normal',
      [1] = 'pre',
    },
%    \end{macrocode}
% \paragraph{boundary (6)}
%    \begin{macrocode}
    boundary = {
      [0] = 'cancel',
      [1] = 'user',
      [2] = 'protrusion',
      [3] = 'word',
    },
%    \end{macrocode}
% \paragraph{disc (7)}
%    \begin{macrocode}
    disc  = {
      [0] = 'discretionary',
      [1] = 'explicit',
      [2] = 'automatic',
      [3] = 'regular',
      [4] = 'first',
      [5] = 'second',
    },
%    \end{macrocode}
%
% \noindent
% Nodes without subtypes:
% \begin{compactitem}
% \item whatsit (8)
% \item local\_par (9)
% \item dir (10)
% \end{compactitem}
%
% \paragraph{math (11)}
%    \begin{macrocode}
    math = {
      [0] = 'beginmath',
      [1] = 'endmath',
    },
%    \end{macrocode}
% \paragraph{glue (12)}
%    \begin{macrocode}
    glue = {
      [0]   = 'userskip',
      [1]   = 'lineskip',
      [2]   = 'baselineskip',
      [3]   = 'parskip',
      [4]   = 'abovedisplayskip',
      [5]   = 'belowdisplayskip',
      [6]   = 'abovedisplayshortskip',
      [7]   = 'belowdisplayshortskip',
      [8]   = 'leftskip',
      [9]   = 'rightskip',
      [10]  = 'topskip',
      [11]  = 'splittopskip',
      [12]  = 'tabskip',
      [13]  = 'spaceskip',
      [14]  = 'xspaceskip',
      [15]  = 'parfillskip',
      [16]  = 'mathskip',
      [17]  = 'thinmuskip',
      [18]  = 'medmuskip',
      [19]  = 'thickmuskip',
      [98]  = 'conditionalmathskip',
      [99]  = 'muglue',
      [100] = 'leaders',
      [101] = 'cleaders',
      [102] = 'xleaders',
      [103] = 'gleaders',
    },
%    \end{macrocode}
% \paragraph{kern (13)}
%    \begin{macrocode}
    kern = {
      [0] = 'fontkern',
      [1] = 'userkern',
      [2] = 'accentkern',
      [3] = 'italiccorrection',
    },
%    \end{macrocode}
%
% \noindent
% Nodes without subtypes:
% \begin{compactitem}
% \item penalty (14)
% \item unset (15)
% \item style (16)
% \item choice (17)
% \end{compactitem}
%
% \paragraph{noad (18)}
%    \begin{macrocode}
    noad = {
      [0] = 'ord',
      [1] = 'opdisplaylimits',
      [2] = 'oplimits',
      [3] = 'opnolimits',
      [4] = 'bin',
      [5] = 'rel',
      [6] = 'open',
      [7] = 'close',
      [8] = 'punct',
      [9] = 'inner',
      [10] = 'under',
      [11] = 'over',
      [12] = 'vcenter',
    },
%    \end{macrocode}
% \paragraph{radical (19)}
%    \begin{macrocode}
    radical = {
      [0] = 'radical',
      [1] = 'uradical',
      [2] = 'uroot',
      [3] = 'uunderdelimiter',
      [4] = 'uoverdelimiter',
      [5] = 'udelimiterunder',
      [6] = 'udelimiterover',
    },
%    \end{macrocode}
%
% \noindent
% Nodes without subtypes:
% \begin{compactitem}
% \item fraction (20)
% \end{compactitem}
%
% \paragraph{accent (21)}
%    \begin{macrocode}
    accent = {
      [0] = 'bothflexible',
      [1] = 'fixedtop',
      [2] = 'fixedbottom',
      [3] = 'fixedboth',
    },
%    \end{macrocode}
% \paragraph{fence (22)}
%    \begin{macrocode}
    fence = {
      [0] = 'unset',
      [1] = 'left',
      [2] = 'middle',
      [3] = 'right',
    },
%    \end{macrocode}
%
% \noindent
% Nodes without subtypes:
% \begin{compactitem}
% \item math\_char (23)
% \item sub\_box (24)
% \item sub\_mlist (25)
% \item math\_text\_char (26)
% \item delim (27)
% \item margin\_kern (28)
% \end{compactitem}
%
% \paragraph{glyph (29)}
%    \begin{macrocode}
    glyph = {
      [0] = 'character',
      [1] = 'ligature',
      [2] = 'ghost',
      [3] = 'left',
      [4] = 'right',
    },
%    \end{macrocode}
%
% \noindent
% Nodes without subtypes:
% \begin{compactitem}
% \item align\_record (30)
% \item pseudo\_file (31)
% \item pseudo\_line (32)
% \item page\_insert (33)
% \item split\_insert (34)
% \item expr\_stack (35)
% \item nested\_list (36)
% \item span (37)
% \item attribute (38)
% \item glue\_spec (39)
% \item attribute\_list (40)
% \item temp (41)
% \item align\_stack (42)
% \item movement\_stack (43)
% \item if\_stack (44)
% \item unhyphenated (45)
% \item hyphenated (46)
% \item delta (47)
% \item passive (48)
% \item shape (49)
% \end{compactitem}
%    \begin{macrocode}
  }

  subtypes.whatsit = node.whatsits()

  local out = ''
  if subtypes[typ] and subtypes[typ][n.subtype] then
    out = subtypes[typ][n.subtype]

    if options.verbosity > 1 then
      out = out .. template.type_id(n.subtype)
    end

    return out
  else
    return tostring(n.subtype)
  end

  assert(false)
end
%    \end{macrocode}
%
% \subsubsection{template}
%
%    \begin{macrocode}
function template.color_code(code)
  return string.char(27) .. '[' .. tostring(code) .. 'm'
end
%    \end{macrocode}
%
%    \begin{macrocode}
function template.color(color, mode)
  if options.color ~= 'colored' then
    return ''
  end

  local out = ''
  local code = ''

  if mode == 'bright' then
    out = template.color_code(1)
  elseif mode == 'dim' then
    out = template.color_code(2)
  end

  if color == 'reset' then code = 0
  elseif color == 'red' then code = 31
  elseif color == 'green' then code = 32
  elseif color == 'yellow' then code = 33
  elseif color == 'blue' then code = 34
  elseif color == 'magenta' then code = 35
  elseif color == 'cyan' then code = 36
  else code = 37 end

  return out .. template.color_code(code)

end
%    \end{macrocode}
%
%    \begin{macrocode}
function template.key_value(key, value)
  return template.color('yellow') .. key .. ': ' .. template.color('white') .. value .. '; ' .. template.color('reset')
end
%    \end{macrocode}
%
%    \begin{macrocode}
function template.length(input)
  input = tonumber(input)
  input = input / 2^16
  input = math.floor((input * 10^2) + 0.5) / (10^2)
  return string.format('%gpt', input)
end
%    \end{macrocode}
%
%    \begin{macrocode}
function template.char(input)
  return string.format('%q', unicode.utf8.char(input))
end
%    \end{macrocode}
%
% t = type
%    \begin{macrocode}
function template.type(t, id)
  local out = ''
  out = template.type_color(t) .. string.upper(t)

  if options.verbosity > 1 then
    out = out .. template.type_id(id)
  end

  return out .. template.color('reset')  .. ' '
end
%    \end{macrocode}
%
%    \begin{macrocode}
function template.type_id(id)
  return '[' .. tostring(id) .. ']'
end
%    \end{macrocode}
%
%    \begin{macrocode}
function template.branch(connection_type, connection_state, last)
  local c = connection_type
  local s = connection_state
  local l = last
  if c == 'list' and s == 'stop' and l == false then
    return ' '
  elseif c == 'field' and s == 'stop' and l == false then
    return ' '
  elseif c == 'list' and s == 'continue' and l == false then
    return '│ '
  elseif c == 'field' and s == 'continue' and l == false then
    return '║ '
  elseif c == 'list' and s == 'continue' and l == true then
    return '├─'
  elseif c == 'field' and s == 'continue' and l == true then
    return '╠═'
  elseif c == 'list' and s == 'stop' and l == true then
    return '└─'
  elseif c == 'field' and s == 'stop' and l == true then
    return '╚═'
  end
end
%    \end{macrocode}

%    \begin{macrocode}
function template.branches(level, connection_type)
  local out = ''

  for i = 1, level - 1  do
    out = out .. template.branch('list', nodetree.state[i]['list'], false)
    out = out .. template.branch('field', nodetree.state[i]['field'], false)
  end
%    \end{macrocode}
% Format the last branches
%    \begin{macrocode}
  if connection_type == 'list' then
    out = out .. template.branch('list', nodetree.state[level]['list'], true)
  else
    out = out .. template.branch('list', nodetree.state[level]['list'], false)
    out = out .. template.branch('field', nodetree.state[level]['field'], true)
  end

  return out
end
%    \end{macrocode}
%
%    \begin{macrocode}
function template.init_node_colors()
  template.node_colors = {
    hlist          = template.color('red'),
    vlist          = template.color('green'),
    rule           = template.color('yellow'),
    ins            = template.color('blue'),
    mark           = template.color('magenta'),
    adjust         = template.color('cyan'),
    boundary       = template.color('red', 'bright'),
    disc           = template.color('green', 'bright'),
    whatsit        = template.color('yellow', 'bright'),
    local_par      = template.color('blue', 'bright'),
    dir            = template.color('magenta', 'bright'),
    math           = template.color('cyan', 'bright'),
    glue           = template.color('red'),
    kern           = template.color('green'),
    penalty        = template.color('yellow'),
    unset          = template.color('blue'),
    style          = template.color('magenta'),
    choice         = template.color('cyan'),
    noad           = template.color('red'),
    radical        = template.color('green'),
    fraction       = template.color('yellow'),
    accent         = template.color('blue'),
    fence          = template.color('magenta'),
    math_char      = template.color('cyan'),
    sub_box        = template.color('red', 'bright'),
    sub_mlist      = template.color('green', 'bright'),
    math_text_char = template.color('yellow', 'bright'),
    delim          = template.color('blue', 'bright'),
    margin_kern    = template.color('magenta', 'bright'),
    glyph          = template.color('cyan', 'bright'),
    align_record   = template.color('red'),
    pseudo_file    = template.color('green'),
    pseudo_line    = template.color('yellow'),
    page_insert    = template.color('blue'),
    split_insert   = template.color('magenta'),
    expr_stack     = template.color('cyan'),
    nested_list    = template.color('red'),
    span           = template.color('green'),
    attribute      = template.color('yellow'),
    glue_spec      = template.color('magenta'),
    attribute_list = template.color('cyan'),
    temp           = template.color('magenta'),
    align_stack    = template.color('red', 'bright'),
    movement_stack = template.color('green', 'bright'),
    if_stack       = template.color('yellow', 'bright'),
    unhyphenated   = template.color('magenta', 'bright'),
    hyphenated     = template.color('cyan', 'bright'),
    delta          = template.color('red'),
    passive        = template.color('green'),
    shape          = template.color('yellow'),
  }
end
%    \end{macrocode}
%
%    \begin{macrocode}
function template.type_color(id)
  if not template.node_colors then
    template.init_node_colors()
  end
  return template.node_colors[id]
end
%    \end{macrocode}
%
%    \begin{macrocode}
function template.print(text)
  if options.print == 'start' then
    print(text)
  end
end
%    \end{macrocode}
%
% \subsubsection{nodetree}
%
%    \begin{macrocode}
function nodetree.format_field(head, field)
  local out = ''

  if not head[field] or head[field] == 0 then
    return ''
  end

  if options.verbosity < 2 and field == 'prev' or field == 'next' or field == 'id' or field == 'attr' then
    return ''
  end

  if field == 'prev' or field == 'next' then
    out = nodex.node_id(head[field])
  elseif field == 'subtype' then
    out = nodex.subtype(head)
  elseif field == 'width' or field == 'height' or field == 'depth' then
    out = template.length(head[field])
  elseif field == 'char' then
    out = template.char(head[field])
  else
    out = tostring(head[field])
  end

  return template.key_value(field, out)
end
%    \end{macrocode}
%
% |level| is a integer beginning with 1. The variable |connection_type|
% is a string, which can be either |list| or |field|. The variable
% |connection_state| is a string, which can be either |continue| or
% |stop|.
%    \begin{macrocode}
function nodetree.set_state(level, connection_type, connection_state)
  if not nodetree.state[level] then
    nodetree.state[level] = {}
  end
  nodetree.state[level][connection_type] = connection_state
end
%    \end{macrocode}
%
%    \begin{macrocode}
function nodetree.analyze_node(head, level)
  local out = {}
  local connection_state

  if head.id == node.id('whatsit')
    and head.subtype == node.subtype('user_defined')
    and head.user_id == options.user_id and not options.global == 'true' then
    options.print = head.value
  end

  out = template.type(node.type(head.id), head.id)

  if options.verbosity > 1 then
    out = out .. template.key_value('no', nodex.node_id(head))
  end

  local tmp = {}
  local r = {} -- recurison

  fields = node.fields(head.id, head.subtype)

  for field_id, field_name in pairs(fields) do
    if field_name ~= 'next' and
      field_name ~= 'prev' and
      field_name ~= 'attr' and
      node.is_node(head[field_name]) then
      r[field_name] = head[field_name]
    else
      tmp[#tmp + 1] = nodetree.format_field(head, field_name)
    end
  end

  if head.next then
    connection_state = 'continue'
  else
    connection_state = 'stop'
  end

  nodetree.set_state(level, 'list', connection_state)
  template.print(template.branches(level, 'list') .. out .. table.concat(tmp, ''))

  local max = 0
  for _ in pairs(r) do
    max = max + 1
  end

  local count = 0
  for field_name, recursion_node in pairs(r) do
    count = count + 1
    if count == max then
      connection_state = 'stop'
    else
      connection_state = 'continue'
    end

    nodetree.set_state(level, 'field', connection_state)
    template.print(template.branches(level, 'field') .. field_name .. ':')
    nodetree.analyze_list(recursion_node, level + 1)
  end

end
%    \end{macrocode}
%
%    \begin{macrocode}
function nodetree.analyze_list(head, level)
  while head do
    nodetree.analyze_node(head, level)
    head = head.next
  end
end
%    \end{macrocode}
%
%    \begin{macrocode}
function nodetree.analyze(head)
  template.print('\n')
  template.print(base.get_callback() .. ':\n│')
  nodetree.analyze_list(head, 1)
  return head
end
%    \end{macrocode}
%
%    \begin{macrocode}
function base.set_option(key, value)
  if not options then
    options = {}
  end
  options[key] = value
end
%    \end{macrocode}

%    \begin{macrocode}
function base.get_option(key)
  if not options then
    options = {}
  end
  if options[key] then
    return options[key]
  end
end
%    \end{macrocode}
%
%    \begin{macrocode}
function base.set_default_options()
  local defaults = {
    verbosity = 1,
    channel = 'term',
    callback = 'postlinebreak',
    engine = 'luatex',
    color = 'colored',
    user_id = 43192,
  }
  if not options then
    options = {}
  end
  for key, value in pairs(defaults) do
    if not options[key] then
      options[key] = value
    end
  end
  options.verbosity = tonumber(options.verbosity)
end
%    \end{macrocode}

%    \begin{macrocode}
local callbacks = {}
function callbacks.post_linebreak_filter(head, groupcode)
  template.print('post_linebreak_filter')
  if groupcode then
    template.print('groundcode: ' .. groupcode)
  end
  nodetree.analyze_list(head, 1)
  return true
end
%    \end{macrocode}
%
%    \begin{macrocode}
function callbacks.vpack_filter(head, groupcode, size, packtype, direction, attributelist)
  template.print('vpack_filter')
  if groupcode then
    template.print('groundcode: ' .. groupcode)
  end
  nodetree.analyze_list(head, 1)
  return true
end
%    \end{macrocode}
%
%    \begin{macrocode}
function callbacks.hpack_filter(head, groupcode, size, packtype, direction, attributelist)
  template.print('hpack_filter')
  if groupcode then
    template.print('groundcode: ' .. groupcode)
  end
  nodetree.analyze_list(head, 1)
  return true
end
%    \end{macrocode}

% \subsubsection{base}
%
%    \begin{macrocode}
function base.get_callback_name(alias)
  if alias == 'prelinebreak' then return 'pre_linebreak_filter'
  elseif alias == 'linebreak' then return 'linebreak_filter'
  elseif alias == 'postlinebreak' then return 'post_linebreak_filter'
  elseif alias == 'hpack' then return 'hpack_filter'
  elseif alias == 'vpack' then return 'vpack_filter'
  elseif alias == 'hyphenate' then return 'hyphenate'
  elseif alias == 'ligaturing' then return 'ligaturing'
  elseif alias == 'kerning' then return 'kerning'
  elseif alias == 'mhlist' then return 'mlist_to_hlist'
  else return 'post_linebreak_filter'
  end
end
%    \end{macrocode}
%
%    \begin{macrocode}
function base.register(cb)
  print(cb)
  if options.engine == 'lualatex' then
    luatexbase.add_to_callback(cb, callbacks[cb], 'nodetree')
  else
    id, error = callback.register(cb, callbacks[cb])
  end
end
%    \end{macrocode}
%
%    \begin{macrocode}
function base.register_callbacks()
  for alias in string.gmatch(options.callback, '([^,]+)') do
    base.register(base.get_callback_name(alias))
  end
end
%    \end{macrocode}
%
%    \begin{macrocode}
function base.unregister(cb)
  if options.engine == 'lualatex' then
    luatexbase.remove_from_callback(cb, 'nodetree')
  else
    id, error = callback.register(cb, nil)
  end
end
%    \end{macrocode}
%
%    \begin{macrocode}
function base.unregister_callbacks()
  for alias in string.gmatch(options.callback, '([^,]+)') do
    base.unregister(base.get_callback_name(alias))
  end
end
%    \end{macrocode}
%
%    \begin{macrocode}
function base.execute()
  local c = base.get_callback()
  if options.engine == 'lualatex' then
    luatexbase.add_to_callback(c, callbacks.post_linebreak_filter, 'nodetree')
  else
    id, error = callback.register(c, callbacks.post_linebreak_filter)
  end
end
%    \end{macrocode}
%
%    \begin{macrocode}
function base.analyze(head)
  nodetree.analyze_list(head, 1)
end
%    \end{macrocode}
%
%    \begin{macrocode}
base.marker = nodex.create_marker
return base
%    \end{macrocode}
% \iffalse
%</luamain>
% \fi
%
% \Finale
\endinput

\directlua{
  nodetree.set_option('engine', 'lualatex')
}
%    \end{macrocode}
%
%    \begin{macrocode}
\RequirePackage{kvoptions}
%    \end{macrocode}
%
%    \begin{macrocode}
\SetupKeyvalOptions{
  family=NT,
  prefix=NT@
}
%    \end{macrocode}
%
%    \begin{macrocode}
\DeclareStringOption[term]{channel}
\define@key{NT}{channel}[]{\nodetreeoption[channel]{#1}}
%    \end{macrocode}
%
%    \begin{macrocode}
\DeclareStringOption[postlinebreak]{callback}
\define@key{NT}{callback}[]{\nodetreeoption[callback]{#1}}
%    \end{macrocode}
%
%    \begin{macrocode}
\DeclareStringOption[1]{verbosity}
\define@key{NT}{verbosity}[]{\nodetreeoption[verbosity]{#1}}
%    \end{macrocode}
%
%    \begin{macrocode}
\DeclareStringOption[colored]{color}
\define@key{NT}{color}[]{\nodetreeoption[color]{#1}}
%    \end{macrocode}
%
%    \begin{macrocode}
\DeclareStringOption[1]{unit}
\define@key{NT}{unit}[]{\nodetreeoption[unit]{#1}}
%    \end{macrocode}
%
%    \begin{macrocode}
\DeclareStringOption[1]{decimalplaces}
\define@key{NT}{decimalplaces}[]{\nodetreeoption[decimalplaces]{#1}}
%    \end{macrocode}
% Never load “heavy” packages like |mdframed| in default debug mode. The
% are way to slow.
%    \begin{macrocode}
\newif\ifdocumentationmode%
\documentationmodefalse%
\DeclareVoidOption{documentationmode}{%
  \RequirePackage{xcolor,mdframed,expl3}%
  \nodetreeoption[callback]{}%
  \documentationmodetrue%
}
%    \end{macrocode}
%
%    \begin{macrocode}
\ProcessKeyvalOptions*
\directlua{
  nodetree.set_default_options()
  nodetree.register_callbacks()
}
%    \end{macrocode}
%
% \begin{macro}{\nodetreeset}
%    \begin{macrocode}
\newcommand{\nodetreeset}[1]{\setkeys{nodetree}{#1}}
%    \end{macrocode}
% \end{macro}
%
% \begin{macro}{\NT@theme}
%    \begin{macrocode}
\ifdocumentationmode
\ExplSyntaxOn
\def\NT@theme#1{
  \str_case:nn{#1}{
    {terminalapp}{
      \definecolor{NTblack}{RGB}{0,0,0}
      \definecolor{NTred}{RGB}{194,54,33}
      \definecolor{NTgreen}{RGB}{37,188,36}
      \definecolor{NTyellow}{RGB}{173,173,39}
      \definecolor{NTblue}{RGB}{73,46,225}
      \definecolor{NTmagenta}{RGB}{211,56,211}
      \definecolor{NTcyan}{RGB}{51,187,200}
      \definecolor{NTgray}{RGB}{203,204,205}
      \definecolor{NTblackbright}{RGB}{129,131,131}
      \definecolor{NTredbright}{RGB}{252,57,31}
      \definecolor{NTgreenbright}{RGB}{49,231,34}
      \definecolor{NTyellowbright}{RGB}{234,236,35}
      \definecolor{NTbluebright}{RGB}{88,51,255}
      \definecolor{NTmagentabright}{RGB}{249,53,248}
      \definecolor{NTcyanbright}{RGB}{20,240,240}
      \definecolor{NTgraybright}{RGB}{233,235,235}
    }
    {xterm}{
      \definecolor{NTblack}{RGB}{0,0,0}
      \definecolor{NTred}{RGB}{205,0,0}
      \definecolor{NTgreen}{RGB}{0,205,0}
      \definecolor{NTyellow}{RGB}{205,205,0}
      \definecolor{NTblue}{RGB}{0,0,238}
      \definecolor{NTmagenta}{RGB}{205,0,205}
      \definecolor{NTcyan}{RGB}{0,205,205}
      \definecolor{NTgray}{RGB}{229,229,229}
      \definecolor{NTblackbright}{RGB}{127,127,127}
      \definecolor{NTredbright}{RGB}{255,0,0}
      \definecolor{NTgreenbright}{RGB}{0,255,0}
      \definecolor{NTyellowbright}{RGB}{255,255,0}
      \definecolor{NTbluebright}{RGB}{92,92,255}
      \definecolor{NTmagentabright}{RGB}{255,0,255}
      \definecolor{NTcyanbright}{RGB}{0,255,255}
      \definecolor{NTgraybright}{RGB}{255,255,255}
    }
    {smyck}{
      \definecolor{NTblack}{HTML}{212121}
      \definecolor{NTred}{HTML}{C75646}
      \definecolor{NTgreen}{HTML}{8EB33B}
      \definecolor{NTyellow}{HTML}{D0B03C}
      \definecolor{NTblue}{HTML}{72B3CC}
      \definecolor{NTmagenta}{HTML}{C8A0D1}
      \definecolor{NTcyan}{HTML}{218693}
      \definecolor{NTgray}{HTML}{B0B0B0}
      \definecolor{NTblackbright}{HTML}{5D5D5D}
      \definecolor{NTredbright}{HTML}{E09690}
      \definecolor{NTgreenbright}{HTML}{CDEE69}
      \definecolor{NTyellowbright}{HTML}{FFE377}
      \definecolor{NTbluebright}{HTML}{9CD9F0}
      \definecolor{NTmagentabright}{HTML}{FBB1F9}
      \definecolor{NTcyanbright}{HTML}{77DFD8}
      \definecolor{NTgraybright}{HTML}{F7F7F7}
    }
  }
}
\ExplSyntaxOff
%    \end{macrocode}
% \end{macro}
%
% \begin{environment}{nodetreeexample}
%    \begin{macrocode}
\newenvironment{nodetreeexample}{%
  \tiny%
  \setmonofont{DejaVu Sans Mono}%
  \ttfamily%
  \NT@theme{smyck}%
  \let\NT@c\textcolor%
  \def\NT@w{\hspace{0.5em}}
  \color{white}%
  \setlength{\parindent}{0pt}%
  \setlength{\parskip}{-0.85pt}%
  \makeatletter%
}{%
  \makeatother
}
\makeatletter
\NT@theme{smyck}
\surroundwithmdframed[linecolor=black,backgroundcolor=NTblack,fontcolor=white]{nodetreeexample}
\makeatother
\fi
%    \end{macrocode}
% \end{environment}
%
% \iffalse
%</package>
%<*luamain>
% \fi
%
% \makeatletter
% \c@CodelineNo 0 \relax
% \makeatother
%
% \subsection{The file \tt{nodetree.lua}}
%
%    \begin{macrocode}
local nodex = {}
%    \end{macrocode}
%
%    \begin{macrocode}
local tpl = {}
%    \end{macrocode}
%
%    \begin{macrocode}
local tree = {}
%    \end{macrocode}
%
% Nodes in Lua\TeX{} are connected. The nodetree view distinguishs
% between the |list| and |field| connections.
%
% \begin{itemize}
%  \item |list|: Nodes, which are double connected by |next| and
%        |previous| fields.
%  \item |field|: Connections to nodes by other fields than |next| and
%        |previous| fields, e. g. |head|, |pre|.
% \end{itemize}
%
% The lua table named |tree.state| holds state values for the current
% tree item.
%
% \begin{code}
%  tree.state:
%    - 1:
%      - list: continue
%      - field: stop
%    - 2:
%      - list: continue
%      - field: stop
% \end{code}
%    \begin{macrocode}
tree.state = {}
%    \end{macrocode}
%
%    \begin{macrocode}
local callbacks = {}
%    \end{macrocode}
%
%    \begin{macrocode}
local base = {}
%    \end{macrocode}
%
%    \begin{macrocode}
local options = {}
%    \end{macrocode}
%
%    \begin{macrocode}
local output_file = {}
%    \end{macrocode}
%
% \subsubsection{nodex --- Extend the node library}
%
% Get the node id form, e. g.:
% \begin{code}
% <node    nil <    172 >    nil : hlist 2>
% \end{code}
%    \begin{macrocode}
function nodex.node_id(n)
  return string.gsub(tostring(n), '^<node%s+%S+%s+<%s+(%d+).*', '%1')
end
%    \end{macrocode}
%
%    \begin{macrocode}
function nodex.subtype(n)
  local typ = node.type(n.id)
  local subtypes = {
%    \end{macrocode}
% \paragraph{hlist (0)}
%    \begin{macrocode}
    hlist = {
      [0] = 'unknown',
      [1] = 'line',
      [2] = 'box',
      [3] = 'indent',
      [4] = 'alignment',
      [5] = 'cell',
      [6] = 'equation',
      [7] = 'equationnumber',
    },
%    \end{macrocode}
% \paragraph{vlist (1)}
%    \begin{macrocode}
    vlist = {
      [0] = 'unknown',
      [4] = 'alignment',
      [5] = 'cell',
    },
%    \end{macrocode}
% \paragraph{rule (2)}
%    \begin{macrocode}
    rule = {
      [0] = 'unknown',
      [1] = 'box',
      [2] = 'image',
      [3] = 'empty',
      [4] = 'user',
    },
%    \end{macrocode}
%
% \noindent
% Nodes without subtypes:
% \begin{compactitem}
% \item ins (3)
% \item mark (4)
% \end{compactitem}
%    \begin{macrocode}
%    \end{macrocode}
% \paragraph{adjust (5)}
%    \begin{macrocode}
    adjust = {
      [0] = 'normal',
      [1] = 'pre',
    },
%    \end{macrocode}
% \paragraph{boundary (6)}
%    \begin{macrocode}
    boundary = {
      [0] = 'cancel',
      [1] = 'user',
      [2] = 'protrusion',
      [3] = 'word',
    },
%    \end{macrocode}
% \paragraph{disc (7)}
%    \begin{macrocode}
    disc  = {
      [0] = 'discretionary',
      [1] = 'explicit',
      [2] = 'automatic',
      [3] = 'regular',
      [4] = 'first',
      [5] = 'second',
    },
%    \end{macrocode}
%
% \noindent
% Nodes without subtypes:
% \begin{compactitem}
% \item whatsit (8)
% \item local\_par (9)
% \item dir (10)
% \end{compactitem}
%
% \paragraph{math (11)}
%    \begin{macrocode}
    math = {
      [0] = 'beginmath',
      [1] = 'endmath',
    },
%    \end{macrocode}
% \paragraph{glue (12)}
%    \begin{macrocode}
    glue = {
      [0]   = 'userskip',
      [1]   = 'lineskip',
      [2]   = 'baselineskip',
      [3]   = 'parskip',
      [4]   = 'abovedisplayskip',
      [5]   = 'belowdisplayskip',
      [6]   = 'abovedisplayshortskip',
      [7]   = 'belowdisplayshortskip',
      [8]   = 'leftskip',
      [9]   = 'rightskip',
      [10]  = 'topskip',
      [11]  = 'splittopskip',
      [12]  = 'tabskip',
      [13]  = 'spaceskip',
      [14]  = 'xspaceskip',
      [15]  = 'parfillskip',
      [16]  = 'mathskip',
      [17]  = 'thinmuskip',
      [18]  = 'medmuskip',
      [19]  = 'thickmuskip',
      [98]  = 'conditionalmathskip',
      [99]  = 'muglue',
      [100] = 'leaders',
      [101] = 'cleaders',
      [102] = 'xleaders',
      [103] = 'gleaders',
    },
%    \end{macrocode}
% \paragraph{kern (13)}
%    \begin{macrocode}
    kern = {
      [0] = 'fontkern',
      [1] = 'userkern',
      [2] = 'accentkern',
      [3] = 'italiccorrection',
    },
%    \end{macrocode}
%
% \noindent
% Nodes without subtypes:
% \begin{compactitem}
% \item penalty (14)
% \item unset (15)
% \item style (16)
% \item choice (17)
% \end{compactitem}
%
% \paragraph{noad (18)}
%    \begin{macrocode}
    noad = {
      [0] = 'ord',
      [1] = 'opdisplaylimits',
      [2] = 'oplimits',
      [3] = 'opnolimits',
      [4] = 'bin',
      [5] = 'rel',
      [6] = 'open',
      [7] = 'close',
      [8] = 'punct',
      [9] = 'inner',
      [10] = 'under',
      [11] = 'over',
      [12] = 'vcenter',
    },
%    \end{macrocode}
% \paragraph{radical (19)}
%    \begin{macrocode}
    radical = {
      [0] = 'radical',
      [1] = 'uradical',
      [2] = 'uroot',
      [3] = 'uunderdelimiter',
      [4] = 'uoverdelimiter',
      [5] = 'udelimiterunder',
      [6] = 'udelimiterover',
    },
%    \end{macrocode}
%
% \noindent
% Nodes without subtypes:
% \begin{compactitem}
% \item fraction (20)
% \end{compactitem}
%
% \paragraph{accent (21)}
%    \begin{macrocode}
    accent = {
      [0] = 'bothflexible',
      [1] = 'fixedtop',
      [2] = 'fixedbottom',
      [3] = 'fixedboth',
    },
%    \end{macrocode}
% \paragraph{fence (22)}
%    \begin{macrocode}
    fence = {
      [0] = 'unset',
      [1] = 'left',
      [2] = 'middle',
      [3] = 'right',
    },
%    \end{macrocode}
%
% \noindent
% Nodes without subtypes:
% \begin{compactitem}
% \item math\_char (23)
% \item sub\_box (24)
% \item sub\_mlist (25)
% \item math\_text\_char (26)
% \item delim (27)
% \item margin\_kern (28)
% \end{compactitem}
%
% \paragraph{glyph (29)}
%    \begin{macrocode}
    glyph = {
      [0] = 'character',
      [1] = 'ligature',
      [2] = 'ghost',
      [3] = 'left',
      [4] = 'right',
    },
%    \end{macrocode}
%
% \noindent
% Nodes without subtypes:
% \begin{compactitem}
% \item align\_record (30)
% \item pseudo\_file (31)
% \item pseudo\_line (32)
% \item page\_insert (33)
% \item split\_insert (34)
% \item expr\_stack (35)
% \item nested\_list (36)
% \item span (37)
% \item attribute (38)
% \item glue\_spec (39)
% \item attribute\_list (40)
% \item temp (41)
% \item align\_stack (42)
% \item movement\_stack (43)
% \item if\_stack (44)
% \item unhyphenated (45)
% \item hyphenated (46)
% \item delta (47)
% \item passive (48)
% \item shape (49)
% \end{compactitem}
%    \begin{macrocode}
  }
  subtypes.whatsit = node.whatsits()
  local out = ''
  if subtypes[typ] and subtypes[typ][n.subtype] then
    out = subtypes[typ][n.subtype]
    if options.verbosity > 1 then
      out = out .. tpl.type_id(n.subtype)
    end
    return out
  else
    return tostring(n.subtype)
  end
  assert(false)
end
%    \end{macrocode}
%
% \subsubsection{tpl --- Template function}
%
%    \begin{macrocode}
function tpl.underscore(string)
  if options.channel == 'tex' then
    return string.gsub(string, '_', '\\_')
  else
    return string
  end
end
%    \end{macrocode}

%    \begin{macrocode}
function tpl.escape(string)
  if options.channel == 'tex' then
    return string.gsub(string, [[\]], [[\string\]])
  else
    return string
  end
end
%    \end{macrocode}
%
%    \begin{macrocode}
function tpl.round(number)
  local mult = 10^(options.decimalplaces or 0)
  return math.floor(number * mult + 0.5) / mult
end
%    \end{macrocode}
%
%    \begin{macrocode}
function tpl.whitespace(count)
  local whitespace, out = '', ''
  if options.channel == 'tex' then
    whitespace = '\\NT@w'
  else
    whitespace = ' '
  end
  if not count then
    count = 1
  end
  for i = 1, count do
    out = out .. whitespace
  end
  return out
end
%    \end{macrocode}
%
%    \begin{macrocode}
function tpl.length(input)
  input = tonumber(input)
  input = input / tex.sp('1' .. options.unit)
  return string.format('%g%s', tpl.round(input), options.unit)
end
%    \end{macrocode}
%
%    \begin{macrocode}
function tpl.fill(number, order, field)
  if order ~= nil and order ~= 0 then
    if field == 'stretch' then
      out = '+'
    else
      out = '-'
    end
    return out .. string.format(
      '%gfi%s', number / 2^16,
      string.rep('l', order - 1)
    )
  else
    return tpl.length(number)
  end
end
%    \end{macrocode}
%
%    \begin{macrocode}
tpl.node_colors = {
  hlist = {'red', 'bright'},
  vlist = {'green', 'bright'},
  rule = {'blue', 'bright'},
  ins = {'blue'},
  mark = {'magenta'},
  adjust = {'cyan'},
  boundary = {'red', 'bright'},
  disc = {'green', 'bright'},
  whatsit = {'yellow', 'bright'},
  local_par = {'blue', 'bright'},
  dir = {'magenta', 'bright'},
  math = {'cyan', 'bright'},
  glue = {'magenta', 'bright'},
  kern = {'green', 'bright'},
  penalty = {'yellow', 'bright'},
  unset = {'blue'},
  style = {'magenta'},
  choice = {'cyan'},
  noad = {'red'},
  radical = {'green'},
  fraction = {'yellow'},
  accent = {'blue'},
  fence = {'magenta'},
  math_char = {'cyan'},
  sub_box = {'red', 'bright'},
  sub_mlist = {'green', 'bright'},
  math_text_char = {'yellow', 'bright'},
  delim = {'blue', 'bright'},
  margin_kern = {'magenta', 'bright'},
  glyph = {'cyan', 'bright'},
  align_record = {'red'},
  pseudo_file = {'green'},
  pseudo_line = {'yellow'},
  page_insert = {'blue'},
  split_insert = {'magenta'},
  expr_stack = {'cyan'},
  nested_list = {'red'},
  span = {'green'},
  attribute = {'yellow'},
  glue_spec = {'magenta'},
  attribute_list = {'cyan'},
  temp = {'magenta'},
  align_stack = {'red', 'bright'},
  movement_stack = {'green', 'bright'},
  if_stack = {'yellow', 'bright'},
  unhyphenated = {'magenta', 'bright'},
  hyphenated = {'cyan', 'bright'},
  delta = {'red'},
  passive = {'green'},
  shape = {'yellow'},
}
%    \end{macrocode}
%
%    \begin{macrocode}
function tpl.color_code(code)
  return string.char(27) .. '[' .. tostring(code) .. 'm'
end
%    \end{macrocode}
%
% \begin{code}
% local colors = {
%     -- attributes
%     reset = 0,
%     clear = 0,
%     bright = 1,
%     dim = 2,
%     underscore = 4,
%     blink = 5,
%     reverse = 7,
%     hidden = 8,
%
%     -- foreground
%     black = 30,
%     red = 31,
%     green = 32,
%     yellow = 33,
%     blue = 34,
%     magenta = 35,
%     cyan = 36,
%     white = 37,
%
%     -- background
%     onblack = 40,
%     onred = 41,
%     ongreen = 42,
%     onyellow = 43,
%     onblue = 44,
%     onmagenta = 45,
%     oncyan = 46,
%     onwhite = 47,
% }
% \end{code}
%    \begin{macrocode}
function tpl.color(color, mode, background)
  if options.color ~= 'colored' then
    return ''
  end
%    \end{macrocode}
%
%    \begin{macrocode}
  local out = ''
  local code = ''
%    \end{macrocode}
%
%    \begin{macrocode}
  if mode == 'bright' then
    out = tpl.color_code(1)
  elseif mode == 'dim' then
    out = tpl.color_code(2)
  end
%    \end{macrocode}
%
%    \begin{macrocode}
  if not background then
    if color == 'reset' then code = 0
    elseif color == 'red' then code = 31
    elseif color == 'green' then code = 32
    elseif color == 'yellow' then code = 33
    elseif color == 'blue' then code = 34
    elseif color == 'magenta' then code = 35
    elseif color == 'cyan' then code = 36
    else code = 37 end
  else
    if color == 'black' then code = 40
    elseif color == 'red' then code = 41
    elseif color == 'green' then code = 42
    elseif color == 'yellow' then code = 43
    elseif color == 'blue' then code = 44
    elseif color == 'magenta' then code = 45
    elseif color == 'cyan' then code = 46
    elseif color == 'white' then code = 47
    else code = 40 end
  end
  return out .. tpl.color_code(code)
end
%    \end{macrocode}
%
%    \begin{macrocode}
function tpl.color_tex(color, mode, background)
  if not mode then mode = '' end
  return 'NT' .. color .. mode
end
%    \end{macrocode}
%
%    \begin{macrocode}
function tpl.colored_string(string, color, mode, background)
  if options.channel == 'tex' then
    return '\\NT@c{' ..
      tpl.color_tex(color, mode, background) ..
      '}{' ..
      string ..
      '}'
  else
   return tpl.color(color, mode, background) .. string .. tpl.color('reset')
  end
end
%    \end{macrocode}
%
%    \begin{macrocode}
function tpl.key_value(key, value)
  if options.channel == 'tex' then
    key = tpl.underscore(key)
  end
  local out = tpl.colored_string(key .. ':', 'yellow')
  if value then
    out = out .. ' ' .. value .. '; '
  end
  return out
end

%    \end{macrocode}
%
%    \begin{macrocode}
function tpl.char(input)
  input = string.format('%q', unicode.utf8.char(input))
  if options.channel == 'tex' then
    input = tpl.escape(input)
  end
  return input
end
%    \end{macrocode}
%
%    \begin{macrocode}
function tpl.type(type, id)
  local out = ''
  if options.channel == 'tex' then
    out = tpl.underscore(type)
  else
    out = type
  end
  out = string.upper(out)
  if options.verbosity > 1 then
    out = out .. tpl.type_id(id)
  end
  return tpl.colored_string(
    out .. tpl.whitespace(),
    tpl.node_colors[type][1],
    tpl.node_colors[type][2]
  )
end
%    \end{macrocode}
%
%    \begin{macrocode}
function tpl.callback_variable(variable_name, variable)
  if variable ~= nil and variable ~= '' then
    tpl.print(
      tpl.underscore(variable_name) .. ': ' ..
      tostring(variable) ..
      tpl.line_end()
    )
  end
end
%    \end{macrocode}
%
%    \begin{macrocode}
function tpl.line(length)
  local out = ''
  if length == 'long' then
    out = '------------------------------------------'
  else
    out = '-----------------------'
  end
    return out .. tpl.line_end()
end
%    \end{macrocode}
%
%    \begin{macrocode}
function tpl.new_line()
  if options.channel == 'tex' then
    return '\\par\n'
  else
    return '\n'
  end
end
%    \end{macrocode}
%
%    \begin{macrocode}
function tpl.line_begin()
  return ''
  -- return tpl.whitespace(2)
end
%    \end{macrocode}
%
%    \begin{macrocode}
function tpl.line_end(count)
  local out = ''
  if not count then
    count = 1
  end
  for i = 1, count do
    out = out .. tpl.new_line()
  end
  return out
end
%    \end{macrocode}
%
%    \begin{macrocode}
function tpl.callback(callback_name, variables)
  tpl.print(
    tpl.line_end(2) ..
    'Callback: ' ..
    tpl.colored_string(tpl.underscore(callback_name), 'red', '', true) ..
    tpl.line_end()
  )
  if variables then
    for name, value in pairs(variables) do
      if value ~= nil and value ~= '' then
        tpl.print(
          '- ' ..
          tpl.underscore(name) ..
          ': ' ..
          tostring(value) ..
          tpl.line_end()
        )
      end
    end
  end
  tpl.print(tpl.line('long'))
end
%    \end{macrocode}
%
%    \begin{macrocode}
function tpl.type_id(id)
  return '[' .. tostring(id) .. ']'
end
%    \end{macrocode}
%
%    \begin{macrocode}
function tpl.branch(connection_type, connection_state, last)
  local c = connection_type
  local s = connection_state
  local l = last
  if c == 'list' and s == 'stop' and l == false then
    return tpl.whitespace(2)
  elseif c == 'field' and s == 'stop' and l == false then
    return tpl.whitespace(2)
  elseif c == 'list' and s == 'continue' and l == false then
    return '│' .. tpl.whitespace()
  elseif c == 'field' and s == 'continue' and l == false then
    return '║' .. tpl.whitespace()
  elseif c == 'list' and s == 'continue' and l == true then
    return '├─'
  elseif c == 'field' and s == 'continue' and l == true then
    return '╠═'
  elseif c == 'list' and s == 'stop' and l == true then
    return '└─'
  elseif c == 'field' and s == 'stop' and l == true then
    return '╚═'
  end
end
%    \end{macrocode}
%
%    \begin{macrocode}
function tpl.branches(level, connection_type)
  local out = ''
  for i = 1, level - 1  do
    out = out .. tpl.branch('list', tree.state[i]['list'], false)
    out = out .. tpl.branch('field', tree.state[i]['field'], false)
  end
%    \end{macrocode}
% Format the last branches
%    \begin{macrocode}
  if connection_type == 'list' then
    out = out .. tpl.branch('list', tree.state[level]['list'], true)
  else
    out = out .. tpl.branch('list', tree.state[level]['list'], false)
    out = out .. tpl.branch('field', tree.state[level]['field'], true)
  end
  return out
end
%    \end{macrocode}
%
%    \begin{macrocode}
function tpl.print(text)
  if options.channel == 'log' or options.channel == 'tex' then
    output_file:write(text)
  else
    io.write(text)
  end
end
%    \end{macrocode}
%
% \subsubsection{tree --- Build the node tree}
%
%    \begin{macrocode}
function tree.format_field(head, field)
  local out = ''
%    \end{macrocode}
%
%    \begin{macrocode}
  if not head[field] or head[field] == 0 then
    return ''
  end
%    \end{macrocode}
%
%    \begin{macrocode}
  if options.verbosity < 2 and
    -- glyph
    field == 'font' or
    field == 'left' or
    field == 'right' or
    field == 'uchyph' or
    -- hlist
    field == 'dir' or
    field == 'glue_order' or
    field == 'glue_sign' or
    field == 'glue_set' or
    -- glue
    field == 'stretch_order' then
    return ''
  elseif options.verbosity < 3 and
    field == 'prev' or
    field == 'next' or
    field == 'id'
  then
    return ''
  end
%    \end{macrocode}
%
%    \begin{macrocode}
  if field == 'prev' or field == 'next' then
    out = nodex.node_id(head[field])
  elseif field == 'subtype' then
    out = tpl.underscore(nodex.subtype(head))
  elseif
    field == 'width' or
    field == 'height' or
    field == 'depth' or
    field == 'kern' or
    field == 'shift' then
    out = tpl.length(head[field])
  elseif field == 'char' then
    out = tpl.char(head[field])
  elseif field == 'glue_set' then
    out = tpl.round(head[field])
  elseif field == 'stretch' or field == 'shrink' then
    out = tpl.fill(head[field], head[field .. '_order'], field)
  else
    out = tostring(head[field])
  end
%    \end{macrocode}
%
%    \begin{macrocode}
  return tpl.key_value(field, out)
end
%    \end{macrocode}
%
% |level| is a integer beginning with 1. The variable |connection_type|
% is a string, which can be either |list| or |field|. The variable
% |connection_state| is a string, which can be either |continue| or
% |stop|.
%    \begin{macrocode}
function tree.set_state(level, connection_type, connection_state)
  if not tree.state[level] then
    tree.state[level] = {}
  end
  tree.state[level][connection_type] = connection_state
end
%    \end{macrocode}
%
%    \begin{macrocode}
function tree.analyze_fields(fields, level)
  local max = 0
  local connection_state = ''
  for _ in pairs(fields) do
    max = max + 1
  end
  local count = 0
  for field_name, recursion_node in pairs(fields) do
    count = count + 1
    if count == max then
      connection_state = 'stop'
    else
      connection_state = 'continue'
    end
    tree.set_state(level, 'field', connection_state)
    tpl.print(tpl.line_begin() .. tpl.branches(level, 'field') .. tpl.key_value(field_name) .. tpl.line_end())
    tree.analyze_list(recursion_node, level + 1)
  end
end
%    \end{macrocode}
%
%    \begin{macrocode}
function tree.analyze_node(head, level)
  local connection_state
  local out = ''
  if head.next then
    connection_state = 'continue'
  else
    connection_state = 'stop'
  end
  tree.set_state(level, 'list', connection_state)
  out = tpl.branches(level, 'list')
    .. tpl.type(node.type(head.id), head.id)
  if options.verbosity > 1 then
    out = out .. tpl.key_value('no', nodex.node_id(head))
  end
%    \end{macrocode}
%
%    \begin{macrocode}
  local fields = {}
  for field_id, field_name in pairs(node.fields(head.id, head.subtype)) do
    if field_name ~= 'next' and
      field_name ~= 'prev' and
      node.is_node(head[field_name]) then
      fields[field_name] = head[field_name]
    else
      out = out .. tree.format_field(head, field_name)
    end
  end
%    \end{macrocode}
%
%    \begin{macrocode}
  tpl.print(tpl.line_begin() .. out .. tpl.line_end())
  tree.analyze_fields(fields, level)
end
%    \end{macrocode}
%
%    \begin{macrocode}
function tree.analyze_list(head, level)
  while head do
    tree.analyze_node(head, level)
    head = head.next
  end
end
%    \end{macrocode}
%
%    \begin{macrocode}
function tree.analyze_callback(head)
  tree.analyze_list(head, 1)
  tpl.print(tpl.line('short') .. tpl.line_end())
end
%    \end{macrocode}
%
% \subsubsection{callbacks --- Callback wrapper}
%
%    \begin{macrocode}
function callbacks.contribute_filter(extrainfo)
  tpl.callback('contribute_filter', {extrainfo = extrainfo})
  return true
end
%    \end{macrocode}
%
%    \begin{macrocode}
function callbacks.buildpage_filter(extrainfo)
  tpl.callback('buildpage_filter', {extrainfo = extrainfo})
  return true
end
%    \end{macrocode}
%
%    \begin{macrocode}
function callbacks.pre_linebreak_filter(head, groupcode)
  tpl.callback('pre_linebreak_filter', {groupcode = groupcode})
  tree.analyze_callback(head)
  return true
end
%    \end{macrocode}
%
%    \begin{macrocode}
function callbacks.linebreak_filter(head, is_display)
  tpl.callback('linebreak_filter', {is_display = is_display})
  tree.analyze_callback(head)
  return true
end
%    \end{macrocode}
%
% TODO: Fix return values, page output
%    \begin{macrocode}
function callbacks.append_to_vlist_filter(head, locationcode, prevdepth, mirrored)
  local variables = {
    locationcode = locationcode,
    prevdepth = prevdepth,
    mirrored = mirrored,
  }
  tpl.callback('append_to_vlist_filter', variables)
  tree.analyze_callback(head)
  return true
end
%    \end{macrocode}
%
%    \begin{macrocode}
function callbacks.post_linebreak_filter(head, groupcode)
  tpl.callback('post_linebreak_filter', {groupcode = groupcode})
  tree.analyze_callback(head)
  return true
end
%    \end{macrocode}
%
%    \begin{macrocode}
function callbacks.hpack_filter(head, groupcode, size, packtype, direction, attributelist)
  local variables = {
    groupcode = groupcode,
    size = size,
    packtype = packtype,
    direction = direction,
    attributelist = attributelist,
  }
  tpl.callback('hpack_filter', variables)
  tree.analyze_callback(head)
  return true
end
%    \end{macrocode}
%
%    \begin{macrocode}
function callbacks.vpack_filter(head, groupcode, size, packtype, maxdepth, direction, attributelist)
  local variables = {
    groupcode = groupcode,
    size = size,
    packtype = packtype,
    maxdepth = tpl.length(maxdepth),
    direction = direction,
    attributelist = attributelist,
  }
  tpl.callback('vpack_filter', variables)
  tree.analyze_callback(head)
  return true
end
%    \end{macrocode}
%
%    \begin{macrocode}
function callbacks.hpack_quality(incident, detail, head, first, last)
  local variables = {
    incident = incident,
    detail = detail,
    first = first,
    last = last,
  }
  tpl.callback('hpack_quality', variables)
  tree.analyze_callback(head)
end
%    \end{macrocode}
%
%    \begin{macrocode}
function callbacks.vpack_quality(incident, detail, head, first, last)
  local variables = {
    incident = incident,
    detail = detail,
    first = first,
    last = last,
  }
  tpl.callback('vpack_quality', variables)
  tree.analyze_callback(head)
end
%    \end{macrocode}
%
%    \begin{macrocode}
function callbacks.process_rule(head, width, height)
  local variables = {
    width = width,
    height = height,
  }
  tpl.callback('process_rule', variables)
  tree.analyze_callback(head)
  return true
end
%    \end{macrocode}
%
%    \begin{macrocode}
function callbacks.pre_output_filter(head, groupcode, size, packtype, maxdepth, direction)
  local variables = {
    groupcode = groupcode,
    size = size,
    packtype = packtype,
    maxdepth = maxdepth,
    direction = direction,
  }
  tpl.callback('pre_output_filter', variables)
  tree.analyze_callback(head)
  return true
end
%    \end{macrocode}
%
%    \begin{macrocode}
function callbacks.hyphenate(head, tail)
  tpl.callback('hyphenate')
  tpl.print('head:')
  tree.analyze_callback(head)
  tpl.print('tail:')
  tree.analyze_callback(tail)
end
%    \end{macrocode}
%
%    \begin{macrocode}
function callbacks.ligaturing(head, tail)
  tpl.callback('ligaturing')
  tpl.print('head:')
  tree.analyze_callback(head)
  tpl.print('tail:')
  tree.analyze_callback(tail)
end
%    \end{macrocode}
%
%    \begin{macrocode}
function callbacks.kerning(head, tail)
  tpl.callback('kerning')
  tpl.print('head:')
  tree.analyze_callback(head)
  tpl.print('tail:')
  tree.analyze_callback(tail)
end
%    \end{macrocode}
%
%    \begin{macrocode}
function callbacks.insert_local_par(local_par, location)
  tpl.callback('insert_local_par', {location = location})
  tree.analyze_callback(local_par)
  return true
end
%    \end{macrocode}
%
%    \begin{macrocode}
function callbacks.mlist_to_hlist(head, display_type, need_penalties)
  local variables = {
    display_type = display_type,
    need_penalties = need_penalties,
  }
  tpl.callback('mlist_to_hlist', variables)
  tree.analyze_callback(head)
  return node.mlist_to_hlist(head, display_type, need_penalties)
end
%    \end{macrocode}
%
% \subsubsection{base --- Exported base functions}
%
%    \begin{macrocode}
function base.normalize_options()
  options.verbosity = tonumber(options.verbosity)
  options.decimalplaces = tonumber(options.decimalplaces)
end
%    \end{macrocode}
%
%    \begin{macrocode}
function base.set_default_options()
  local defaults = {
    verbosity = 1,
    callback = 'postlinebreak',
    engine = 'luatex',
    color = 'colored',
    decimalplaces = 2,
    unit = 'pt',
    channel = 'term',
  }
  if not options then
    options = {}
  end
  for key, value in pairs(defaults) do
    if not options[key] then
      options[key] = value
    end
  end
  base.normalize_options()
end
%    \end{macrocode}
%
%    \begin{macrocode}
function base.set_option(key, value)
  if not options then
    options = {}
  end
  options[key] = value
  base.normalize_options()
end
%    \end{macrocode}
%
%    \begin{macrocode}
function base.get_option(key)
  if not options then
    options = {}
  end
  if options[key] then
    return options[key]
  end
end
%    \end{macrocode}
%
%    \begin{macrocode}
function base.get_callback_name(alias)
  if alias == 'contribute' or alias == 'contributefilter' then
    return 'contribute_filter'
%    \end{macrocode}
%
%    \begin{macrocode}
  elseif alias == 'buildpage' or alias == 'buildpagefilter' then
    return 'buildpage_filter'
%    \end{macrocode}
%
%    \begin{macrocode}
  elseif alias == 'preline' or alias == 'prelinebreakfilter' then
    return 'pre_linebreak_filter'
%    \end{macrocode}
%
%    \begin{macrocode}
  elseif alias == 'line' or alias == 'linebreakfilter' then
    return 'linebreak_filter'
%    \end{macrocode}
%
%    \begin{macrocode}
  elseif alias == 'append' or alias == 'appendtovlistfilter' then
    return 'append_to_vlist_filter'
%    \end{macrocode}
%
%    \begin{macrocode}
  elseif alias == 'postline' or alias == 'postlinebreakfilter' then
    return 'post_linebreak_filter'
%    \end{macrocode}
%
%    \begin{macrocode}
  elseif alias == 'hpack' or alias == 'hpackfilter' then
    return 'hpack_filter'
%    \end{macrocode}
%
%    \begin{macrocode}
  elseif alias == 'vpack' or alias == 'vpackfilter' then
    return 'vpack_filter'
%    \end{macrocode}
%
%    \begin{macrocode}
  elseif alias == 'hpackq' or alias == 'hpackquality' then
    return 'hpack_quality'
%    \end{macrocode}
%
%    \begin{macrocode}
  elseif alias == 'vpackq' or alias == 'vpackquality' then
    return 'vpack_quality'
%    \end{macrocode}
%
%    \begin{macrocode}
  elseif alias == 'process' or alias == 'processrule' then
    return 'process_rule'
%    \end{macrocode}
%
%    \begin{macrocode}
  elseif alias == 'preout' or alias == 'preoutputfilter' then
    return 'pre_output_filter'
%    \end{macrocode}
%
%    \begin{macrocode}
  elseif alias == 'hyph' or alias == 'hyphenate' then
    return 'hyphenate'
%    \end{macrocode}
%
%    \begin{macrocode}
  elseif alias == 'liga' or alias == 'ligaturing' then
    return 'ligaturing'
%    \end{macrocode}
%
%    \begin{macrocode}
  elseif alias == 'kern' or alias == 'kerning' then
   return 'kerning'
%    \end{macrocode}
%
%    \begin{macrocode}
  elseif alias == 'insert' or alias == 'insertlocalpar' then
    return 'insert_local_par'
%    \end{macrocode}
%
%    \begin{macrocode}
  elseif alias == 'mhlist' or alias == 'mlisttohlist' then
    return 'mlist_to_hlist'
%    \end{macrocode}
%
%    \begin{macrocode}
  else
    return 'post_linebreak_filter'
  end
end
%    \end{macrocode}
%
%    \begin{macrocode}
function base.register(cb)
  if options.engine == 'lualatex' then
    luatexbase.add_to_callback(cb, callbacks[cb], 'nodetree')
  else
    id, error = callback.register(cb, callbacks[cb])
  end
end
%    \end{macrocode}
%
%    \begin{macrocode}
function base.register_callbacks()
  if options.channel == 'log' or options.channel == 'tex' then
    output_file = io.open(tex.jobname .. '_nodetree.' .. options.channel, 'a')
  end
  for alias in string.gmatch(options.callback, '([^,]+)') do
    base.register(base.get_callback_name(alias))
  end
end
%    \end{macrocode}
%
%    \begin{macrocode}
function base.unregister(cb)
  if options.engine == 'lualatex' then
    luatexbase.remove_from_callback(cb, 'nodetree')
  else
    id, error = callback.register(cb, nil)
  end
end
%    \end{macrocode}
%
%    \begin{macrocode}
function base.unregister_callbacks()
  for alias in string.gmatch(options.callback, '([^,]+)') do
    base.unregister(base.get_callback_name(alias))
  end
end
%    \end{macrocode}
%
%    \begin{macrocode}
function base.execute()
  local c = base.get_callback()
  if options.engine == 'lualatex' then
    luatexbase.add_to_callback(c, callbacks.post_linebreak_filter, 'nodetree')
  else
    id, error = callback.register(c, callbacks.post_linebreak_filter)
  end
end
%    \end{macrocode}
%
%    \begin{macrocode}
function base.analyze(head)
  tpl.print(tpl.line_end())
  tree.analyze_list(head, 1)
end
%    \end{macrocode}
%
%    \begin{macrocode}
return base
%    \end{macrocode}
% \iffalse
%</luamain>
% \fi
%
% \Finale
\endinput

% \iffalse meta-comment
%
% Copyright (C) 2016-2020 by Josef Friedrich <josef@friedrich.rocks>
% ----------------------------------------------------------------------
% This work may be distributed and/or modified under the conditions of
% the LaTeX Project Public License, either version 1.3 of this license
% or (at your option) any later version.  The latest version of this
% license is in:
%
%   http://www.latex-project.org/lppl.txt
%
% and version 1.3 or later is part of all distributions of LaTeX
% version 2005/12/01 or later.
%
% This work has the LPPL maintenance status `maintained'.
%
% The Current Maintainer of this work is Josef Friedrich.
%
% This work consists of the files nodetree.dtx and nodetree.ins
% and the derived filebase nodetree.sty and nodetree.lua.
%
% \fi
%
% \iffalse
%<*driver>
\ProvidesFile{nodetree.dtx}
%</driver>
%<package>\NeedsTeXFormat{LaTeX2e}[1999/12/01]
%<package>\ProvidesPackage{nodetree}
%<*package>
    [2016/07/18 v1.2 Visualize node lists in a tree view]
%</package>
%<*driver>
\documentclass{ltxdoc}
\usepackage{paralist,fontspec,graphicx,fancyvrb}
\usepackage[
  colorlinks=true,
  linkcolor=red,
  filecolor=red,
  urlcolor=red,
]{hyperref}
\usepackage[theme=molokai,thememode=dark,documentationmode]{nodetree}
\EnableCrossrefs
\CodelineIndex
\RecordChanges

\usepackage{minted}
\usemintedstyle{colorful}
\BeforeBeginEnvironment{minted}{\begin{mdframed}[backgroundcolor=gray!3]}
\AfterEndEnvironment{minted}{\end{mdframed}}
\setminted{
  breaklines=true,
  fontsize=\footnotesize,
}

\def\nodetreelua#1{\texttt{\scantokens{\catcode`\_=12\relax#1}}}

\def\secref#1{(\rightarrow\ \ref{#1})}

\newcommand{\TmpGraphics}[1]{
  \noindent
  \includegraphics[scale=0.4]{graphics/#1}
}


\newcommand{\TmpExample}[1]{
\begin{nodetreeexample}
\input{examples/#1.nttex}
\end{nodetreeexample}
}

\newcommand{\TmpVerbExample}[1]{
\VerbatimInput[frame=single,fontsize=\footnotesize,firstline=4]{examples/#1.tex}
\TmpExample{#1}
}

\DefineVerbatimEnvironment{code}{Verbatim}
{
  frame=single,
  fontsize=\footnotesize,
}

\newcommand{\TmpLuaFunction}[1]{
  \marginpar{%
    \raggedleft%
    \MacroFont%
    \texttt{%
      \scantokens{\catcode`\_=12\relax#1}%
    }%
  }%
}

\begin{document}

\providecommand*{\url}{\texttt}
\GetFileInfo{nodetree.dtx}
\title{The \textsf{nodetree} package}
\author{%
  Josef Friedrich\\%
  \url{josef@friedrich.rocks}\\%
  \href{https://github.com/Josef-Friedrich/nodetree}{github.com/Josef-Friedrich/nodetree}%
}
\date{\fileversion~from \filedate}

\maketitle

\begin{nodetreeexample}[fontsize=\footnotesize]
%!TEX program = lualatex
\documentclass{article}
\usepackage[channel=tex]{nodetree}
\begin{document}
node
\end{document}

\end{nodetreeexample}

\newpage

\tableofcontents

\newpage

%-----------------------------------------------------------------------
% Abstract
%-----------------------------------------------------------------------

\section{Abstract}

|nodetree| is a development package that visualizes the structure of
node lists. |nodetree| shows its debug informations in the consoles’
output when you compile a Lua\TeX{} file. It uses a similar visual
representation for node lists as the UNIX |tree| command does for a
folder tree.

Node lists are the main building blocks of each document generated by
the \TeX{} engine \emph{Lua\TeX}. The package |nodetree| doesn‘t change
the rendered document. The tree view can only be seen when using a
terminal to generate the document.

|nodetree| is inspired by a
\href{https://gist.github.com/pgundlach/556247}
{gist from Patrick Gundlach}.

%-----------------------------------------------------------------------
% Usage
%-----------------------------------------------------------------------

\section{Usage}

The package |nodetree| has three usage scenarios.
It can be used as a standalone Lua module, as a plain Lua\TeX{} or a
Lua\LaTeX{} package.

%%
%
%%

\subsection{As a plain Lua\TeX{} package}

Run |luatex luatex-test.tex| for example to list the nodes using
Lua\TeX{}.

\begin{minted}{latex}
%%
%% This is file `nodetree.tex',
%% generated with the docstrip utility.
%%
%% The original source files were:
%%
%% nodetree.dtx  (with options: `tex')
%% 
%% This is a generated file.
%% 
%% Copyright (C) 2015 by Josef Friedrich <josef@friedrich.rocks>
%% ----------------------------------------------------------------------
%% This work may be distributed and/or modified under the conditions of
%% the LaTeX Project Public License, either version 1.3c of this license
%% or (at your option) any later version. The latest version of this
%% license is in:
%% 
%%   http://www.latex-project.org/lppl.txt
%% 
%% and version 1.3c or later is part of all distributions of LaTeX
%% version 2008/05/05 or later.
%% 
\directlua{
  nodetree = require('nodetree')
  nodetree.set_option('engine', 'luatex')
}
\def\nodetreeoption[#1]#2{
  \directlua{
    nodetree.set_option('#1', '#2')
  }
}
\endinput
%%
%% End of file `nodetree.tex'.

\nodetreeregister{postline}

Lorem ipsum dolor.
\bye
\end{minted}

%%
%
%%

\subsection{As a Lua\LaTeX{} package}

Or run |lualatex lualatex-test.tex| to show a node tree using
Lua\LaTeX{}. In Lua\LaTeX{} you can omit |\nodetreeregister{postline}|.
|\usepackage{nodetree}| registers automatically the
|post_linebreak_filter|. If you don’t want debug the
|post_linebreak_filter| use |\nodetreeunregister{postline}|.

\begin{minted}{latex}
\documentclass{article}
\usepackage{nodetree}

\begin{document}
Lorem ipsum dolor.
\end{document}
\end{minted}

%%
% inside Lua code
%%

\subsection{As a Lua module}

Import the Lua module of the package inside
\mintinline{latex}{\directlua{}}
with this command:
\mintinline{lua}{local nodetree = require('nodetree')}.
Then use the Lua function \mintinline{lua}{nodetree.print(head, options)} to debug nodes
inside your Lua code.

\begin{minted}{lua}
local nodetree = require('nodetree')

local rule1 = node.new('rule')
rule1.width  = 20 * 65536
rule1.height = 10 * 65536
rule1.depth  = 10 * 65536
nodetree.print(vbox)
\end{minted}

The function \mintinline{lua}{nodetree.print()} takes as a second
argument a Lua table to configure the output.

\begin{minted}{lua}
nodetree.print(vbox, { verbosity = 2, unit = 'cm' })
\end{minted}

This are the default options:

\begin{minted}{lua}
options =  {
  verbosity = 1,
  callback = 'postlinebreak',
  engine = 'luatex',
  color = 'colored',
  decimalplaces = 2,
  unit = 'pt',
  channel = 'term'
}
\end{minted}

The following code snippet demonstrates the usage in Lua\TeX{}.
|head| is the current node.

\begin{minted}{latex}
  \directlua{
  local nodetree = require('nodetree')
  local test = function (head)
    nodetree.analyze(head)
  end
  callback.register('post_linebreak_filter', test)
}

Lorem ipsum dolor.
\bye
\end{minted}

This example illustrates how the function has to be applied in
Lua\LaTeX{}.

\begin{minted}{latex}
\documentclass{article}
\usepackage{nodetree}

\begin{document}

\directlua{
  local test = function (head)
    nodetree.analyze(head)
  end
  luatexbase.add_to_callback('post_linebreak_filter', test, 'test')
}

Lorem ipsum dolor.
\end{document}
\end{minted}

%-----------------------------------------------------------------------
% Macros
%-----------------------------------------------------------------------

\section{Macros}

%%
% \nodetreeregister
%%

\subsection{\cmd{\nodetreeregister}}

\DescribeMacro{\nodetreeregister}
\cmd{\nodetreeregister}\marg{callbacks}: The argument \marg{callbacks}
takes a comma separated list of callback aliases as described in
\secref{sec:option-callback}.

%%
% \nodetreeunregister
%%

\subsection{\cmd{\nodetreeunregister}}

\DescribeMacro{\nodetreeunregister}
\cmd{\nodetreeunregister}\marg{callbacks}: The argument \marg{callbacks}
takes a comma separated list of callback aliases as described in
\secref{sec:option-callback}.

%%
% \nodetreeoption
%%

\subsection{\cmd{\nodetreeoption}}

\DescribeMacro{\nodetreeoption}
\cmd{\nodetreeoption}\oarg{option}\marg{value}: \secref{sec:options}
This macro sets the option \oarg{option} to the value \marg{value}.

%%
% \nodetreeset
%%

\subsection{\cmd{\nodetreeset}}

\DescribeMacro{\nodetreeset}
\cmd{\nodetreeset}\marg{kv-options}:
This macro can only be used in Lua\LaTeX{}. \marg{kv-options} are key
value pairs.

\begin{code}
\nodetreeset{color=no,callbacks={hpack,vpack},verbosity=2}
\end{code}

%-----------------------------------------------------------------------
% Options
%-----------------------------------------------------------------------

\section{Options}
\label{sec:options}

%%
% callback
%%

\subsection{Option \texttt{callback}}
\label{sec:option-callback}

The option |callback| is the most important setting of the package. You
have to specify one alias to select the |callback|. Because of the
underscores the callback name contains it can not set by its technical
name (\rightarrow{} Figure \ref{fig:callback}).

This macros process callback options:
\cmd{\nodetreeregister}\marg{callbacks},
\cmd{\nodetreeunregister}\marg{callbacks},
\cmd{\nodetreeset}\marg{callback=<callbacks>} and
\cmd{\usepackage}\oarg{callback=<callbacks>}\marg{nodetree}.

Use commas to specify mulitple callbacks. Avoid using whitespaces:

\begin{code}
\nodetreeregister{preline,line,postline}
\end{code}

Wrap your callback aliases in curly braces for the macro |\nodetreeset|:

\begin{code}
\nodetreeset{callback={preline,line,postline}}
\end{code}

The same applies for the macro |\usepackage|:

\begin{code}
\usepackage{callback={preline,line,postline}}
\end{code}

%%
% Tabular callbacks
%%

\newcommand{\nodetreecallback}[3]{
  \nodetreelua{#1} & \nodetreelua{#2} & \nodetreelua{#3} \\
}

\begin{figure}

\noindent
\begin{tabular}{lll}
\textbf{Alias (short)} & \textbf{Alias (longer)} & \textbf{Callback} \\
\nodetreecallback{contribute}{contributefilter}{contribute_filter}
\nodetreecallback{buildpage}{buildpagefilter}{buildpage_filter}
\nodetreecallback{preline}{prelinebreakfilter}{pre_linebreak_filter}
\nodetreecallback{line}{linebreakfilter}{linebreak_filter}
\nodetreecallback{append}{appendtovlistfilter}{append_to_vlist_filter}
\nodetreecallback{postline}{postlinebreakfilter}{post_linebreak_filter}
\nodetreecallback{hpack}{hpackfilter}{hpack_filter}
\nodetreecallback{vpack}{vpackfilter}{vpack_filter}
\nodetreecallback{hpackq}{hpackquality}{hpack_quality}
\nodetreecallback{vpackq}{vpackquality}{vpack_quality}
\nodetreecallback{process}{processrule}{process_rule}
\nodetreecallback{preout}{preoutputfilter}{pre_output_filter}
\nodetreecallback{hyph}{hyphenate}{hyphenate}
\nodetreecallback{liga}{ligaturing}{ligaturing}
\nodetreecallback{kern}{kerning}{kerning}
\nodetreecallback{insert}{insertlocalpar}{insert_local_par}
\nodetreecallback{mhlist}{mlisttohlist}{mlist_to_hlist}
\end{tabular}

\caption{The callback aliases}
\label{fig:callback}
\end{figure}

%%
% channel
%%

\subsection{Option \texttt{channel}}
\label{sec:option-channel}

You can select the debug output channel with this option. The default
value for the option |channel| is |term| which displays the node tree in
the current terminal. Specify |log| and the package creates a log file
named |jobname_nodetree.log|. |jobname| is the name of your file you
want to debug.

%%
% verbosity
%

\subsection{Option \texttt{verbosity}}

Higher integer values result in a more verbose output. The default value
for this options is |1|. At the moment only verbosity level |2| is
implemented.

\subsubsection{Example \texttt{verbosity=1}}

\nodetreeterminalemulator{examples/option_verbosity-1}

\subsubsection{Example \texttt{verbosity=2}}

\nodetreeterminalemulator{examples/option_verbosity-2}

\subsubsection{Example \texttt{verbosity=3}}

\nodetreeterminalemulator{examples/option_verbosity-3}

%%
% color
%%

\subsection{Option \texttt{color}}

The default option for |color| is |colored|. Use any other string (for
example |none| or |no|) to disable the colored terminal output of the
package.

\begin{code}
\usepackage[color=no]{nodetree}
\end{code}

%%
% unit
%%

\subsection{Option \texttt{unit}}

The option |unit| sets the length unit to display all length values of
the nodes. The default option for |unit| is |pt|. See figure
\ref{fig:fixed-units} and \ref{fig:relative-units} for possible values.

\begin{figure}
\begin{tabular}{lp{10cm}}
\textbf{Unit} &
\textbf{Description} \\

pt &
Point 1/72.27 inch. The conversion to metric units, to two decimal
places, is 1 point = 2.85 mm = 28.45 cm. \\

pc &
Pica, 12 pt \\

in &
Inch, 72.27 pt \\

bp &
Big point, 1/72 inch. This length is the definition of a point in
PostScript and many desktop publishing systems. \\

cm &
Centimeter \\

mm &
Millimeter \\

dd &
Didot point, 1.07 pt \\

cc &
Cicero, 12 dd \\

sp &
Scaled point, 1/65536 pt \\
\end{tabular}
\caption{Fixed units}
\label{fig:fixed-units}
\end{figure}

\begin{figure}
\begin{tabular}{lp{10cm}}
\textbf{Unit} &
\textbf{Description} \\

ex &
x-height of the current font \\

em &
Width of the capital letter M \\
\end{tabular}
\caption{Relative units}
\label{fig:relative-units}
\end{figure}


\nodetreeterminalemulator{examples/option_unit-pt}
\nodetreeterminalemulator{examples/option_unit-sp}
\nodetreeterminalemulator{examples/option_unit-cm}

%%
% decimalplaces
%%

\subsection{Option \texttt{decimalplaces}}

The options |decimalplaces| sets the number of decimal places for some
node fields.

\begin{code}
\nodetreeoption[decimalplaces]{4}
\end{code}

gets

\begin{code}
├─GLYPH char: "a"; width: 5pt; height: 4.3055pt;
\end{code}

If |decimalplaces| is set to |0| only integer values are shown.

\begin{code}
├─GLYPH char: "a"; width: 5pt; height: 4pt;
\end{code}

%%
% theme and thememode
%%

\newcommand{\TmpExampleTheme}[2]{
  \subsubsection{Example \texttt{theme=#1} \texttt{thememode=#2}}
  \nodetreeterminalemulator[theme=#1,thememode=#2]{examples/minimal}
}

\subsection{Option \texttt{theme} and \texttt{thememode}}

% bw
\TmpExampleTheme{bwdark}{dark}
\TmpExampleTheme{bwlight}{light}

% terminalapp
\TmpExampleTheme{terminalapp}{dark}
\TmpExampleTheme{terminalapp}{light}

% xterm
\TmpExampleTheme{xterm}{dark}
\TmpExampleTheme{xterm}{light}

% smyck
\TmpExampleTheme{smyck}{dark}
\TmpExampleTheme{smyck}{light}

% molokai
\TmpExampleTheme{molokai}{dark}
\TmpExampleTheme{molokai}{light}

% monokaisoda
\TmpExampleTheme{monokaisoda}{dark}
\TmpExampleTheme{monokaisoda}{light}

\nodetreereset

%%
% font
%%

\subsection{Option \texttt{font}}

\nodetreeset{fontsize=\small}

\newcommand{\TmpExampleFont}[1]{
  \subsubsection{Example \texttt{font=\{#1\}}}
  \nodetreeterminalemulator[font={#1}]{examples/minimal}
}

\TmpExampleFont{FreeMono}
\TmpExampleFont{Liberation Mono}
\TmpExampleFont{DejaVu Sans Mono}
\TmpExampleFont{Ubuntu Mono}


\nodetreereset

%%
% fontsize
%%

\subsection{Option \texttt{fontsize}}

\string\small

\nodetreeterminalemulator[fontsize=\small]{examples/minimal}

\string\tiny

\nodetreeterminalemulator[fontsize=\tiny]{examples/minimal}

\nodetreereset

%-----------------------------------------------------------------------
% Visual tree structure
%-----------------------------------------------------------------------

\section{Visual tree structure}

%%
% Two different connections
%%

\subsection{Two different connections}

Nodes in Lua\TeX{} are connected. The |nodetree| package distinguishs
between the |list| and |field| connections.

\begin{itemize}
 \item |list|: Nodes, which are double connected by |next| and
       |previous| fields.
 \item |field|: Connections to nodes by other fields than |next| and
       |previous| fields, e. g. |head|, |pre|.
\end{itemize}

%%
% Unicode characters
%%

\subsection{Unicode characters to show the tree view}

\renewcommand{\arraystretch}{1.5}

The package |nodetree| uses the unicode box drawing symbols. Your
default terminal font should contain this characters to obtain the tree
view. Eight box drawing characters are necessary.


{
\fontspec{DejaVu Sans Mono}
\noindent
\begin{tabular}{lcl}
\textbf{Code} & \textbf{Character} & \textbf{Name} \\
U+2500 & ─ & BOX DRAWINGS LIGHT HORIZONTAL \\
U+2502 & │ & BOX DRAWINGS LIGHT VERTICAL \\
U+2514 & └ & BOX DRAWINGS LIGHT UP AND RIGHT \\
U+251C & ├ & BOX DRAWINGS LIGHT VERTICAL AND RIGHT \\
U+2550 & ═ & BOX DRAWINGS DOUBLE HORIZONTAL \\
U+2551 & ║ & BOX DRAWINGS DOUBLE VERTICAL \\
U+255A & ╚ & BOX DRAWINGS DOUBLE UP AND RIGHT \\
U+2560 & ╠ & BOX DRAWINGS DOUBLE VERTICAL AND RIGHT \\
\end{tabular}
}

\noindent
For |list| connections \emph{light} characters are shown.

{
\setmonofont{DejaVu Sans Mono}
\begin{code}
│ │
│ ├─list1
│ └─list2
└─list3
\end{code}
}

\noindent
|field| connections are visialized by \emph{Double} characters.

{
\setmonofont{DejaVu Sans Mono}
\begin{code}
║ ║
║ ╠═field1
║ ╚═field2
╚═field3
\end{code}
}

%-----------------------------------------------------------------------
% Examples
%-----------------------------------------------------------------------

\section{Examples}

%%
% packagename
%%

\subsection{The node list of the package name}

\TmpVerbExample{packagename}

%%
% math
%%

\subsection{The node list of a mathematical formula}

\TmpVerbExample{math}

%%
% ligatures
%%

\subsection{The node list of the word \emph{Office}}

The characters \emph{ffi} are deeply nested in a discretionary node.

\TmpVerbExample{ligatures}

%-----------------------------------------------------------------------
% Node types
%-----------------------------------------------------------------------

\subsection{Node types}

\newcommand{\TmpNodeTypeSub}[4]{
  \subsubsection{Type: #1(#2) Subtype: #3(#4)}
  \TmpVerbExample{#2#1#4#3}
}

\newcommand{\TmpNodeType}[2]{
  \subsubsection{Type: #1(#2)}
  \TmpVerbExample{#2#1}
}

\TmpNodeTypeSub{hlist}{0}{line}{1}
\TmpNodeTypeSub{hlist}{0}{box}{2}
\TmpNodeTypeSub{hlist}{0}{indent}{3}
\TmpNodeType{vlist}{1}
\TmpNodeType{rule}{2}
\TmpNodeType{mark}{4}
\TmpNodeTypeSub{disc}{7}{discretionary}{0}
\TmpNodeTypeSub{disc}{7}{regular}{3}
\TmpNodeTypeSub{whatsit}{8}{pdfaction}{22}
\TmpNodeTypeSub{whatsit}{8}{pdfcolorstack}{28}
\TmpNodeTypeSub{glue}{12}{baselineskip}{2}
\TmpNodeTypeSub{glue}{12}{parskip}{3}
\TmpNodeTypeSub{glue}{12}{spaceskip}{13}
\TmpNodeTypeSub{glue}{12}{leaders}{100}
\TmpNodeTypeSub{glue}{12}{cleaders}{101}
\TmpNodeTypeSub{glue}{12}{xleaders}{102}
\TmpNodeTypeSub{kern}{13}{userkern}{0}
\TmpNodeTypeSub{kern}{13}{fontkern}{1}
\TmpNodeTypeSub{kern}{13}{accentkern}{2}
\TmpNodeTypeSub{kern}{13}{italiccorrection}{3}
\TmpNodeType{penalty}{14}
\TmpNodeType{glyph}{29}
\TmpNodeType{attribute}{38}
\TmpNodeType{attributelist}{40}

%-----------------------------------------------------------------------
% Index
%-----------------------------------------------------------------------

  \DocInput{nodetree.dtx}
  \pagebreak
  \PrintChanges
  \pagebreak
  \PrintIndex
\end{document}
%</driver>
% \fi
%
% \CheckSum{0}
%
% \CharacterTable
%  {Upper-case    \A\B\C\D\E\F\G\H\I\J\K\L\M\N\O\P\Q\R\S\T\U\V\W\X\Y\Z
%   Lower-case    \a\b\c\d\e\f\g\h\i\j\k\l\m\n\o\p\q\r\s\t\u\v\w\x\y\z
%   Digits        \0\1\2\3\4\5\6\7\8\9
%   Exclamation   \!     Double quote  \"     Hash (number) \#
%   Dollar        \$     Percent       \%     Ampersand     \&
%   Acute accent  \'     Left paren    \(     Right paren   \)
%   Asterisk      \*     Plus          \+     Comma         \,
%   Minus         \-     Point         \.     Solidus       \/
%   Colon         \:     Semicolon     \;     Less than     \<
%   Equals        \=     Greater than  \>     Question mark \?
%   Commercial at \@     Left bracket  \[     Backslash     \\
%   Right bracket \]     Circumflex    \^     Underscore    \_
%   Grave accent  \`     Left brace    \{     Vertical bar  \|
%   Right brace   \}     Tilde         \~}
%
%
% \changes{v0.1}{2015/06/16}{Converted to DTX file}
% \changes{v1.0}{2016/07/07}{Inital release}
% \changes{v1.1}{2016/07/13}{Fix the registration of same callbacks}
% \changes{v1.2}{2016/07/18}{Fix difference between README.md in the upload and that from nodetree.dtx}
%
% \DoNotIndex{\newcommand,\newenvironment,\def,\directlua}
%
% \StopEventually{}
% \pagebreak
% \section{Implementation}
%
% \iffalse
%<*tex>
% \fi
% \MacroTopsep = 10pt plus 2pt minus 2pt
% \MacrocodeTopsep = 10pt plus 1.2pt minus 1pt
% \makeatletter
% \c@CodelineNo 25 \relax
% \makeatother
%
% \subsection{The file \tt{nodetree.tex}}
%
%    \begin{macrocode}
\directlua{
  nodetree = require('nodetree')
  nodetree.set_option('engine', 'luatex')
}
%    \end{macrocode}
%
% \begin{macro}{\nodetreeoption}
%    \begin{macrocode}
\def\nodetreeoption[#1]#2{
  \directlua{
    nodetree.set_option('#1', '#2')
  }
}
%    \end{macrocode}
% \end{macro}
%
% \begin{macro}{\nodetreeregister}
%    \begin{macrocode}
\def\nodetreeregister#1{
  \directlua{
    nodetree.set_option('callback', '#1')
    nodetree.register_callbacks()
  }
}
%    \end{macrocode}
% \end{macro}
%
% \begin{macro}{\nodetreeunregister}
%    \begin{macrocode}
\def\nodetreeunregister#1{
  \directlua{
    nodetree.set_option('callback', '#1')
    nodetree.unregister_callbacks()
  }
}
%    \end{macrocode}
% \end{macro}
%
% \iffalse
%</tex>
%<*package>
% \fi
% \makeatletter
% \c@CodelineNo 25 \relax
% \makeatother
%
% \subsection{The file \tt{nodetree.sty}}
%
%    \begin{macrocode}
% \iffalse meta-comment
%
% Copyright (C) 2015 by Josef Friedrich <josef@friedrich.rocks>
% ----------------------------------------------------------------------
% This work may be distributed and/or modified under the conditions of
% the LaTeX Project Public License, either version 1.3 of this license
% or (at your option) any later version.  The latest version of this
% license is in:
%
%   http://www.latex-project.org/lppl.txt
%
% and version 1.3 or later is part of all distributions of LaTeX
% version 2005/12/01 or later.
%
% This work has the LPPL maintenance status `maintained'.
%
% The Current Maintainer of this work is Josef Friedrich.
%
% This work consists of the files nodetree.dtx and nodetree.ins
% and the derived filebase nodetree.sty and nodetree.lua.
%
% \fi
%
% \iffalse
%<*driver>
\ProvidesFile{nodetree.dtx}
%</driver>
%<package>\NeedsTeXFormat{LaTeX2e}[1999/12/01]
%<package>\ProvidesPackage{nodetree}
%<*package>
    [2015/11/13 Package to debug node lists used by LuaTeX]
%</package>
%<*driver>
\documentclass{ltxdoc}
\usepackage{hyperref,paralist}
\EnableCrossrefs
\CodelineIndex
\RecordChanges
\begin{document}

\providecommand*{\url}{\texttt}
\GetFileInfo{nodetree.dtx}
\title{The \textsf{nodetree} package}
\author{%
  Josef Friedrich\\%
  \url{josef@friedrich.rocks}\\%
  \href{https://github.com/Josef-Friedrich/cloze}{github.com/Josef-Friedrich/nodetree}%
}
\date{\fileversion~from \filedate}

\maketitle

\tableofcontents

\section{Option 'channel'}

Value "term"

\begin{verbatim}
\usepackage[channel=term]{nodetree}
\end{verbatim}

Value "log"

\begin{verbatim}
\usepackage[channel=log]{nodetree}
\end{verbatim}

Value "term and log"

\begin{verbatim}
\usepackage[channel={term and log}]{nodetree}
\end{verbatim}

  \DocInput{nodetree.dtx}
  \pagebreak
  \PrintChanges
  \pagebreak
  \PrintIndex
\end{document}
%</driver>
%<*readme>
# nodetree

Inspired by a [gist of Patrick Gundlach](https://gist.github.com/pgundlach/556247).

`nodetree` displays some debug informations of the node list in the
terminal, when you render a Latex document.

```
post_linebreak_filter:
│
├─GLUE subtype: baselineskip; width: 5.06pt;
└─HLIST subtype: line; width: 345pt; height: 6.94pt; dir: TLT; glue_order: 2; glue_sign: 1; glue_set: 304.99993896484;
 ╚═head:
  ├─LOCAL_PAR dir: TLT;
  ├─HLIST subtype: indent; width: 15pt; dir: TLT;
  ├─GLYPH char: "O"; font: 15; left: 2; right: 3; uchyph: 1; width: 7.78pt; height: 6.83pt;
  ├─DISC subtype: regular; penalty: 50;
  │ ╠═post:
  │ ║ └─GLYPH subtype: ghost; char: "\12"; font: 15; width: 5.56pt; height: 6.94pt;
  │ ║  ╚═components:
  │ ║   ├─GLYPH subtype: ligature; char: "f"; font: 15; left: 2; right: 3; uchyph: 1; width: 3.06pt; height: 6.94pt;
  │ ║   └─GLYPH subtype: ligature; char: "i"; font: 15; left: 2; right: 3; uchyph: 1; width: 2.78pt; height: 6.68pt;
  │ ╠═pre:
  │ ║ ├─GLYPH char: "f"; font: 15; left: 2; right: 3; uchyph: 1; width: 3.06pt; height: 6.94pt;
  │ ║ └─GLYPH char: "-"; font: 15; left: 2; right: 3; uchyph: 1; width: 3.33pt; height: 4.31pt;
  │ ╚═replace:
  │  └─GLYPH subtype: ghost; char: "\14"; font: 15; width: 8.33pt; height: 6.94pt;
  │   ╚═components:
  │    ├─GLYPH subtype: ghost; char: "\11"; font: 15; width: 5.83pt; height: 6.94pt;
  │    │ ╚═components:
  │    │  ├─GLYPH subtype: ligature; char: "f"; font: 15; left: 2; right: 3; uchyph: 1; width: 3.06pt; height: 6.94pt;
  │    │  └─GLYPH subtype: ligature; char: "f"; font: 15; left: 2; right: 3; uchyph: 1; width: 3.06pt; height: 6.94pt;
  │    └─GLYPH subtype: ligature; char: "i"; font: 15; left: 2; right: 3; uchyph: 1; width: 2.78pt; height: 6.68pt;
  ├─GLYPH char: "c"; font: 15; left: 2; right: 3; uchyph: 1; width: 4.44pt; height: 4.31pt;
  ├─GLYPH char: "e"; font: 15; left: 2; right: 3; uchyph: 1; width: 4.44pt; height: 4.31pt;
  ├─PENALTY penalty: 10000;
  ├─GLUE subtype: parfillskip; stretch: 65536; stretch_order: 2;
  └─GLUE subtype: rightskip;

```

# UTF8 Box drawing symbols

## Light

```
│ │
│ ├─┤field1: 1pt├┤field2: 1pt│
│ └─
└─
```

## Heavy

```
┃ ┃
┃ ┣━┫field1: 1pt┣┫field2: 1pt┃
┃ ┗━
┗━
```

## Double

```
║ ║
║ ╠═╣field1: 1pt╠╣field2: 1pt║
║ ╚═
╚═
```

%</readme>
% \fi
%
% \CheckSum{0}
%
% \CharacterTable
%  {Upper-case    \A\B\C\D\E\F\G\H\I\J\K\L\M\N\O\P\Q\R\S\T\U\V\W\X\Y\Z
%   Lower-case    \a\b\c\d\e\f\g\h\i\j\k\l\m\n\o\p\q\r\s\t\u\v\w\x\y\z
%   Digits        \0\1\2\3\4\5\6\7\8\9
%   Exclamation   \!     Double quote  \"     Hash (number) \#
%   Dollar        \$     Percent       \%     Ampersand     \&
%   Acute accent  \'     Left paren    \(     Right paren   \)
%   Asterisk      \*     Plus          \+     Comma         \,
%   Minus         \-     Point         \.     Solidus       \/
%   Colon         \:     Semicolon     \;     Less than     \<
%   Equals        \=     Greater than  \>     Question mark \?
%   Commercial at \@     Left bracket  \[     Backslash     \\
%   Right bracket \]     Circumflex    \^     Underscore    \_
%   Grave accent  \`     Left brace    \{     Vertical bar  \|
%   Right brace   \}     Tilde         \~}
%
%
% \changes{v0.1}{2015/06/16}{Converted to DTX file}
% \changes{v1.0}{2015/07/08}{Inital release}
%
% \DoNotIndex{\newcommand,\newenvironment,\def,\directlua}
%
% \StopEventually{}
% \pagebreak
% \section{Implementation}
%
% \iffalse
%<*tex>
% \fi
% \MacroTopsep = 10pt plus 2pt minus 2pt
% \MacrocodeTopsep = 10pt plus 1.2pt minus 1pt
% \makeatletter
% \c@CodelineNo 25 \relax
% \makeatother
%
% \subsection{The file \tt{nodetree.tex}}
%
%    \begin{macrocode}
\directlua{
  nodetree = require('nodetree')
}
\def\nodetreeoption[#1]#2{
  \directlua{
    nodetree.set_option('#1', '#2')
  }
}
\def\nodetreeshowoption#1{
  \directlua{
    tex.print(nodetree.get_option('#1'))
  }
}
\def\nodetreeprocessoptions{
  \nodetreeoption[engine]{luatex}
  \directlua{
    nodetree.set_default_options()
    nodetree.register_callbacks()
  }
}
\def\nodetreeregister{%
  \strut%
  \directlua{nodetree.marker('start')}%
}
\def\nodetreeunregister{%
  \directlua{nodetree.marker('stop')}%
  \strut%
}
%    \end{macrocode}
%    \end{macrocode}
%
% \iffalse
%</tex>
%<*package>
% \fi
% \makeatletter
% \c@CodelineNo 25 \relax
% \makeatother
%
% \subsection{The file \tt{nodetree.sty}}
%
%    \begin{macrocode}
\directlua{
  nodetree = require('nodetree')
  nodetree.set_option('engine', 'lualatex')
}
%    \end{macrocode}
%    \begin{macrocode}
\RequirePackage{kvoptions}
%    \end{macrocode}
%
%    \begin{macrocode}
\SetupKeyvalOptions{
  family=NT,
  prefix=NT@
}
%    \end{macrocode}
%
%    \begin{macrocode}
\def\NT@set@option[#1]#2{%
  \directlua{nodetree.set_option('#1', '#2')}%
}
\DeclareStringOption[colored]{color}
\define@key{NT}{color}[]{\NT@set@option[color]{#1}}
%    \end{macrocode}
%    \begin{macrocode}
\DeclareStringOption[term and log]{channel}
\define@key{NT}{channel}[]{\NT@set@option[channel]{#1}}
%    \end{macrocode}
%
%    \begin{macrocode}
\DeclareStringOption[postlinebreak]{callback}
\define@key{NT}{callback}[]{\NT@set@option[callback]{#1}}
%    \end{macrocode}
%
%    \begin{macrocode}
\DeclareStringOption[1]{verbosity}
\define@key{NT}{verbosity}[]{\NT@set@option[verbosity]{#1}}
%    \end{macrocode}
%
%    \begin{macrocode}
\DeclareVoidOption{global}{
  \NT@set@option[global]{true}
  \NT@set@option[print]{start}
}
%    \end{macrocode}
%
%    \begin{macrocode}
\ProcessKeyvalOptions*
\directlua{
  nodetree.set_default_options()
  nodetree.register_callbacks()
}
%    \end{macrocode}
%
%    \begin{macrocode}
\newcommand{\nodetreeset}[1]{\setkeys{nodetree}{#1}}
%    \end{macrocode}
%
% \begin{environment}{nodetreeenv}
%    \begin{macrocode}
\newenvironment{nodetreeenv}{%
  \strut%
  \directlua{
    nodetree.marker('start')
  }%
}{%
  \directlua{
    nodetree.marker('stop')
  }%
  \strut%
}
%    \end{macrocode}
% \end{environment}
% \iffalse
%</package>
%<*luamain>
% \fi
%
% \makeatletter
% \c@CodelineNo 0 \relax
% \makeatother
%
% \subsection{The file \tt{nodetree.lua}}
%
%
%    \begin{macrocode}
local nodex = {}
%    \end{macrocode}
%
%    \begin{macrocode}
local template = {}
%    \end{macrocode}
%
%    \begin{macrocode}
local nodetree = {}
%    \end{macrocode}
%
% Nodes in Lua\TeX are connected. The nodetree view distinguishs betweens
% the |list| and |field| connections.

% \begin{itemize}
%  \item |list|: Nodes, which are double connected by |next| and
%        |previous| fields.
%  \item |field|: Connections to nodes by other fields than |next| and
%        |previous| fields, e. g. |head|, |pre|.
% \end{itemize}
%
% The lua table named |nodetree| holds states values for the present nodetree
% item.
% \begin{verbatim}
%  nodetree:
%    - 1:
%      - list: continue
%      - field: stop
%    - 2:
%      - list: continue
%      - field: stop
% \end{verbatim}
%    \begin{macrocode}
nodetree.state = {}
%    \end{macrocode}
%
%    \begin{macrocode}
local base = {}
%    \end{macrocode}
%
%    \begin{macrocode}
local options = {}
%    \end{macrocode}
%
% \subsubsection{nodex --- node extended}
%
% Get the node id form, e. g.:
% \begin{verbatim}
% <node    nil <    172 >    nil : hlist 2>
% \end{verbatim}
%    \begin{macrocode}
function nodex.node_id(n)
  return string.gsub(tostring(n), '^<node%s+%S+%s+<%s+(%d+).*', '%1')
end
%    \end{macrocode}
%
%    \begin{macrocode}
function nodex.create_marker(string)
  local marker = node.new('whatsit','user_defined')
  marker.type = 115
  marker.user_id = options.user_id
  marker.value = string
  node.write(marker)
end
%    \end{macrocode}
%
%    \begin{macrocode}
function nodex.subtype(n)
  local typ = node.type(n.id)

  local subtypes = {
%    \end{macrocode}
% \paragraph{hlist (0)}
%    \begin{macrocode}
    hlist = {
      [0] = 'unknown',
      [1] = 'line',
      [2] = 'box',
      [3] = 'indent',
      [4] = 'alignment',
      [5] = 'cell',
      [6] = 'equation',
      [7] = 'equationnumber',
    },
%    \end{macrocode}
% \paragraph{vlist (1)}
%    \begin{macrocode}
    vlist = {
      [0] = 'unknown',
      [4] = 'alignment',
      [5] = 'cell',
    },
%    \end{macrocode}
% \paragraph{rule (2)}
%    \begin{macrocode}
    rule = {
      [0] = 'unknown',
      [1] = 'box',
      [2] = 'image',
      [3] = 'empty',
      [4] = 'user',
    },
%    \end{macrocode}
%
% \noindent
% Nodes without subtypes:
% \begin{compactitem}
% \item ins (3)
% \item mark (4)
% \end{compactitem}
%    \begin{macrocode}
%    \end{macrocode}
% \paragraph{adjust (5)}
%    \begin{macrocode}
    adjust = {
      [0] = 'normal',
      [1] = 'pre',
    },
%    \end{macrocode}
% \paragraph{boundary (6)}
%    \begin{macrocode}
    boundary = {
      [0] = 'cancel',
      [1] = 'user',
      [2] = 'protrusion',
      [3] = 'word',
    },
%    \end{macrocode}
% \paragraph{disc (7)}
%    \begin{macrocode}
    disc  = {
      [0] = 'discretionary',
      [1] = 'explicit',
      [2] = 'automatic',
      [3] = 'regular',
      [4] = 'first',
      [5] = 'second',
    },
%    \end{macrocode}
%
% \noindent
% Nodes without subtypes:
% \begin{compactitem}
% \item whatsit (8)
% \item local\_par (9)
% \item dir (10)
% \end{compactitem}
%
% \paragraph{math (11)}
%    \begin{macrocode}
    math = {
      [0] = 'beginmath',
      [1] = 'endmath',
    },
%    \end{macrocode}
% \paragraph{glue (12)}
%    \begin{macrocode}
    glue = {
      [0]   = 'userskip',
      [1]   = 'lineskip',
      [2]   = 'baselineskip',
      [3]   = 'parskip',
      [4]   = 'abovedisplayskip',
      [5]   = 'belowdisplayskip',
      [6]   = 'abovedisplayshortskip',
      [7]   = 'belowdisplayshortskip',
      [8]   = 'leftskip',
      [9]   = 'rightskip',
      [10]  = 'topskip',
      [11]  = 'splittopskip',
      [12]  = 'tabskip',
      [13]  = 'spaceskip',
      [14]  = 'xspaceskip',
      [15]  = 'parfillskip',
      [16]  = 'mathskip',
      [17]  = 'thinmuskip',
      [18]  = 'medmuskip',
      [19]  = 'thickmuskip',
      [98]  = 'conditionalmathskip',
      [99]  = 'muglue',
      [100] = 'leaders',
      [101] = 'cleaders',
      [102] = 'xleaders',
      [103] = 'gleaders',
    },
%    \end{macrocode}
% \paragraph{kern (13)}
%    \begin{macrocode}
    kern = {
      [0] = 'fontkern',
      [1] = 'userkern',
      [2] = 'accentkern',
      [3] = 'italiccorrection',
    },
%    \end{macrocode}
%
% \noindent
% Nodes without subtypes:
% \begin{compactitem}
% \item penalty (14)
% \item unset (15)
% \item style (16)
% \item choice (17)
% \end{compactitem}
%
% \paragraph{noad (18)}
%    \begin{macrocode}
    noad = {
      [0] = 'ord',
      [1] = 'opdisplaylimits',
      [2] = 'oplimits',
      [3] = 'opnolimits',
      [4] = 'bin',
      [5] = 'rel',
      [6] = 'open',
      [7] = 'close',
      [8] = 'punct',
      [9] = 'inner',
      [10] = 'under',
      [11] = 'over',
      [12] = 'vcenter',
    },
%    \end{macrocode}
% \paragraph{radical (19)}
%    \begin{macrocode}
    radical = {
      [0] = 'radical',
      [1] = 'uradical',
      [2] = 'uroot',
      [3] = 'uunderdelimiter',
      [4] = 'uoverdelimiter',
      [5] = 'udelimiterunder',
      [6] = 'udelimiterover',
    },
%    \end{macrocode}
%
% \noindent
% Nodes without subtypes:
% \begin{compactitem}
% \item fraction (20)
% \end{compactitem}
%
% \paragraph{accent (21)}
%    \begin{macrocode}
    accent = {
      [0] = 'bothflexible',
      [1] = 'fixedtop',
      [2] = 'fixedbottom',
      [3] = 'fixedboth',
    },
%    \end{macrocode}
% \paragraph{fence (22)}
%    \begin{macrocode}
    fence = {
      [0] = 'unset',
      [1] = 'left',
      [2] = 'middle',
      [3] = 'right',
    },
%    \end{macrocode}
%
% \noindent
% Nodes without subtypes:
% \begin{compactitem}
% \item math\_char (23)
% \item sub\_box (24)
% \item sub\_mlist (25)
% \item math\_text\_char (26)
% \item delim (27)
% \item margin\_kern (28)
% \end{compactitem}
%
% \paragraph{glyph (29)}
%    \begin{macrocode}
    glyph = {
      [0] = 'character',
      [1] = 'ligature',
      [2] = 'ghost',
      [3] = 'left',
      [4] = 'right',
    },
%    \end{macrocode}
%
% \noindent
% Nodes without subtypes:
% \begin{compactitem}
% \item align\_record (30)
% \item pseudo\_file (31)
% \item pseudo\_line (32)
% \item page\_insert (33)
% \item split\_insert (34)
% \item expr\_stack (35)
% \item nested\_list (36)
% \item span (37)
% \item attribute (38)
% \item glue\_spec (39)
% \item attribute\_list (40)
% \item temp (41)
% \item align\_stack (42)
% \item movement\_stack (43)
% \item if\_stack (44)
% \item unhyphenated (45)
% \item hyphenated (46)
% \item delta (47)
% \item passive (48)
% \item shape (49)
% \end{compactitem}
%    \begin{macrocode}
  }

  subtypes.whatsit = node.whatsits()

  local out = ''
  if subtypes[typ] and subtypes[typ][n.subtype] then
    out = subtypes[typ][n.subtype]

    if options.verbosity > 1 then
      out = out .. template.type_id(n.subtype)
    end

    return out
  else
    return tostring(n.subtype)
  end

  assert(false)
end
%    \end{macrocode}
%
% \subsubsection{template}
%
%    \begin{macrocode}
function template.color_code(code)
  return string.char(27) .. '[' .. tostring(code) .. 'm'
end
%    \end{macrocode}
%
%    \begin{macrocode}
function template.color(color, mode)
  if options.color ~= 'colored' then
    return ''
  end

  local out = ''
  local code = ''

  if mode == 'bright' then
    out = template.color_code(1)
  elseif mode == 'dim' then
    out = template.color_code(2)
  end

  if color == 'reset' then code = 0
  elseif color == 'red' then code = 31
  elseif color == 'green' then code = 32
  elseif color == 'yellow' then code = 33
  elseif color == 'blue' then code = 34
  elseif color == 'magenta' then code = 35
  elseif color == 'cyan' then code = 36
  else code = 37 end

  return out .. template.color_code(code)

end
%    \end{macrocode}
%
%    \begin{macrocode}
function template.key_value(key, value)
  return template.color('yellow') .. key .. ': ' .. template.color('white') .. value .. '; ' .. template.color('reset')
end
%    \end{macrocode}
%
%    \begin{macrocode}
function template.length(input)
  input = tonumber(input)
  input = input / 2^16
  input = math.floor((input * 10^2) + 0.5) / (10^2)
  return string.format('%gpt', input)
end
%    \end{macrocode}
%
%    \begin{macrocode}
function template.char(input)
  return string.format('%q', unicode.utf8.char(input))
end
%    \end{macrocode}
%
% t = type
%    \begin{macrocode}
function template.type(t, id)
  local out = ''
  out = template.type_color(t) .. string.upper(t)

  if options.verbosity > 1 then
    out = out .. template.type_id(id)
  end

  return out .. template.color('reset')  .. ' '
end
%    \end{macrocode}
%
%    \begin{macrocode}
function template.type_id(id)
  return '[' .. tostring(id) .. ']'
end
%    \end{macrocode}
%
%    \begin{macrocode}
function template.branch(connection_type, connection_state, last)
  local c = connection_type
  local s = connection_state
  local l = last
  if c == 'list' and s == 'stop' and l == false then
    return ' '
  elseif c == 'field' and s == 'stop' and l == false then
    return ' '
  elseif c == 'list' and s == 'continue' and l == false then
    return '│ '
  elseif c == 'field' and s == 'continue' and l == false then
    return '║ '
  elseif c == 'list' and s == 'continue' and l == true then
    return '├─'
  elseif c == 'field' and s == 'continue' and l == true then
    return '╠═'
  elseif c == 'list' and s == 'stop' and l == true then
    return '└─'
  elseif c == 'field' and s == 'stop' and l == true then
    return '╚═'
  end
end
%    \end{macrocode}

%    \begin{macrocode}
function template.branches(level, connection_type)
  local out = ''

  for i = 1, level - 1  do
    out = out .. template.branch('list', nodetree.state[i]['list'], false)
    out = out .. template.branch('field', nodetree.state[i]['field'], false)
  end
%    \end{macrocode}
% Format the last branches
%    \begin{macrocode}
  if connection_type == 'list' then
    out = out .. template.branch('list', nodetree.state[level]['list'], true)
  else
    out = out .. template.branch('list', nodetree.state[level]['list'], false)
    out = out .. template.branch('field', nodetree.state[level]['field'], true)
  end

  return out
end
%    \end{macrocode}
%
%    \begin{macrocode}
function template.init_node_colors()
  template.node_colors = {
    hlist          = template.color('red'),
    vlist          = template.color('green'),
    rule           = template.color('yellow'),
    ins            = template.color('blue'),
    mark           = template.color('magenta'),
    adjust         = template.color('cyan'),
    boundary       = template.color('red', 'bright'),
    disc           = template.color('green', 'bright'),
    whatsit        = template.color('yellow', 'bright'),
    local_par      = template.color('blue', 'bright'),
    dir            = template.color('magenta', 'bright'),
    math           = template.color('cyan', 'bright'),
    glue           = template.color('red'),
    kern           = template.color('green'),
    penalty        = template.color('yellow'),
    unset          = template.color('blue'),
    style          = template.color('magenta'),
    choice         = template.color('cyan'),
    noad           = template.color('red'),
    radical        = template.color('green'),
    fraction       = template.color('yellow'),
    accent         = template.color('blue'),
    fence          = template.color('magenta'),
    math_char      = template.color('cyan'),
    sub_box        = template.color('red', 'bright'),
    sub_mlist      = template.color('green', 'bright'),
    math_text_char = template.color('yellow', 'bright'),
    delim          = template.color('blue', 'bright'),
    margin_kern    = template.color('magenta', 'bright'),
    glyph          = template.color('cyan', 'bright'),
    align_record   = template.color('red'),
    pseudo_file    = template.color('green'),
    pseudo_line    = template.color('yellow'),
    page_insert    = template.color('blue'),
    split_insert   = template.color('magenta'),
    expr_stack     = template.color('cyan'),
    nested_list    = template.color('red'),
    span           = template.color('green'),
    attribute      = template.color('yellow'),
    glue_spec      = template.color('magenta'),
    attribute_list = template.color('cyan'),
    temp           = template.color('magenta'),
    align_stack    = template.color('red', 'bright'),
    movement_stack = template.color('green', 'bright'),
    if_stack       = template.color('yellow', 'bright'),
    unhyphenated   = template.color('magenta', 'bright'),
    hyphenated     = template.color('cyan', 'bright'),
    delta          = template.color('red'),
    passive        = template.color('green'),
    shape          = template.color('yellow'),
  }
end
%    \end{macrocode}
%
%    \begin{macrocode}
function template.type_color(id)
  if not template.node_colors then
    template.init_node_colors()
  end
  return template.node_colors[id]
end
%    \end{macrocode}
%
%    \begin{macrocode}
function template.print(text)
  if options.print == 'start' then
    print(text)
  end
end
%    \end{macrocode}
%
% \subsubsection{nodetree}
%
%    \begin{macrocode}
function nodetree.format_field(head, field)
  local out = ''

  if not head[field] or head[field] == 0 then
    return ''
  end

  if options.verbosity < 2 and field == 'prev' or field == 'next' or field == 'id' or field == 'attr' then
    return ''
  end

  if field == 'prev' or field == 'next' then
    out = nodex.node_id(head[field])
  elseif field == 'subtype' then
    out = nodex.subtype(head)
  elseif field == 'width' or field == 'height' or field == 'depth' then
    out = template.length(head[field])
  elseif field == 'char' then
    out = template.char(head[field])
  else
    out = tostring(head[field])
  end

  return template.key_value(field, out)
end
%    \end{macrocode}
%
% |level| is a integer beginning with 1. The variable |connection_type|
% is a string, which can be either |list| or |field|. The variable
% |connection_state| is a string, which can be either |continue| or
% |stop|.
%    \begin{macrocode}
function nodetree.set_state(level, connection_type, connection_state)
  if not nodetree.state[level] then
    nodetree.state[level] = {}
  end
  nodetree.state[level][connection_type] = connection_state
end
%    \end{macrocode}
%
%    \begin{macrocode}
function nodetree.analyze_node(head, level)
  local out = {}
  local connection_state

  if head.id == node.id('whatsit')
    and head.subtype == node.subtype('user_defined')
    and head.user_id == options.user_id and not options.global == 'true' then
    options.print = head.value
  end

  out = template.type(node.type(head.id), head.id)

  if options.verbosity > 1 then
    out = out .. template.key_value('no', nodex.node_id(head))
  end

  local tmp = {}
  local r = {} -- recurison

  fields = node.fields(head.id, head.subtype)

  for field_id, field_name in pairs(fields) do
    if field_name ~= 'next' and
      field_name ~= 'prev' and
      field_name ~= 'attr' and
      node.is_node(head[field_name]) then
      r[field_name] = head[field_name]
    else
      tmp[#tmp + 1] = nodetree.format_field(head, field_name)
    end
  end

  if head.next then
    connection_state = 'continue'
  else
    connection_state = 'stop'
  end

  nodetree.set_state(level, 'list', connection_state)
  template.print(template.branches(level, 'list') .. out .. table.concat(tmp, ''))

  local max = 0
  for _ in pairs(r) do
    max = max + 1
  end

  local count = 0
  for field_name, recursion_node in pairs(r) do
    count = count + 1
    if count == max then
      connection_state = 'stop'
    else
      connection_state = 'continue'
    end

    nodetree.set_state(level, 'field', connection_state)
    template.print(template.branches(level, 'field') .. field_name .. ':')
    nodetree.analyze_list(recursion_node, level + 1)
  end

end
%    \end{macrocode}
%
%    \begin{macrocode}
function nodetree.analyze_list(head, level)
  while head do
    nodetree.analyze_node(head, level)
    head = head.next
  end
end
%    \end{macrocode}
%
%    \begin{macrocode}
function nodetree.analyze(head)
  template.print('\n')
  template.print(base.get_callback() .. ':\n│')
  nodetree.analyze_list(head, 1)
  return head
end
%    \end{macrocode}
%
%    \begin{macrocode}
function base.set_option(key, value)
  if not options then
    options = {}
  end
  options[key] = value
end
%    \end{macrocode}

%    \begin{macrocode}
function base.get_option(key)
  if not options then
    options = {}
  end
  if options[key] then
    return options[key]
  end
end
%    \end{macrocode}
%
%    \begin{macrocode}
function base.set_default_options()
  local defaults = {
    verbosity = 1,
    channel = 'term',
    callback = 'postlinebreak',
    engine = 'luatex',
    color = 'colored',
    user_id = 43192,
  }
  if not options then
    options = {}
  end
  for key, value in pairs(defaults) do
    if not options[key] then
      options[key] = value
    end
  end
  options.verbosity = tonumber(options.verbosity)
end
%    \end{macrocode}

%    \begin{macrocode}
local callbacks = {}
function callbacks.post_linebreak_filter(head, groupcode)
  template.print('post_linebreak_filter')
  if groupcode then
    template.print('groundcode: ' .. groupcode)
  end
  nodetree.analyze_list(head, 1)
  return true
end
%    \end{macrocode}
%
%    \begin{macrocode}
function callbacks.vpack_filter(head, groupcode, size, packtype, direction, attributelist)
  template.print('vpack_filter')
  if groupcode then
    template.print('groundcode: ' .. groupcode)
  end
  nodetree.analyze_list(head, 1)
  return true
end
%    \end{macrocode}
%
%    \begin{macrocode}
function callbacks.hpack_filter(head, groupcode, size, packtype, direction, attributelist)
  template.print('hpack_filter')
  if groupcode then
    template.print('groundcode: ' .. groupcode)
  end
  nodetree.analyze_list(head, 1)
  return true
end
%    \end{macrocode}

% \subsubsection{base}
%
%    \begin{macrocode}
function base.get_callback_name(alias)
  if alias == 'prelinebreak' then return 'pre_linebreak_filter'
  elseif alias == 'linebreak' then return 'linebreak_filter'
  elseif alias == 'postlinebreak' then return 'post_linebreak_filter'
  elseif alias == 'hpack' then return 'hpack_filter'
  elseif alias == 'vpack' then return 'vpack_filter'
  elseif alias == 'hyphenate' then return 'hyphenate'
  elseif alias == 'ligaturing' then return 'ligaturing'
  elseif alias == 'kerning' then return 'kerning'
  elseif alias == 'mhlist' then return 'mlist_to_hlist'
  else return 'post_linebreak_filter'
  end
end
%    \end{macrocode}
%
%    \begin{macrocode}
function base.register(cb)
  print(cb)
  if options.engine == 'lualatex' then
    luatexbase.add_to_callback(cb, callbacks[cb], 'nodetree')
  else
    id, error = callback.register(cb, callbacks[cb])
  end
end
%    \end{macrocode}
%
%    \begin{macrocode}
function base.register_callbacks()
  for alias in string.gmatch(options.callback, '([^,]+)') do
    base.register(base.get_callback_name(alias))
  end
end
%    \end{macrocode}
%
%    \begin{macrocode}
function base.unregister(cb)
  if options.engine == 'lualatex' then
    luatexbase.remove_from_callback(cb, 'nodetree')
  else
    id, error = callback.register(cb, nil)
  end
end
%    \end{macrocode}
%
%    \begin{macrocode}
function base.unregister_callbacks()
  for alias in string.gmatch(options.callback, '([^,]+)') do
    base.unregister(base.get_callback_name(alias))
  end
end
%    \end{macrocode}
%
%    \begin{macrocode}
function base.execute()
  local c = base.get_callback()
  if options.engine == 'lualatex' then
    luatexbase.add_to_callback(c, callbacks.post_linebreak_filter, 'nodetree')
  else
    id, error = callback.register(c, callbacks.post_linebreak_filter)
  end
end
%    \end{macrocode}
%
%    \begin{macrocode}
function base.analyze(head)
  nodetree.analyze_list(head, 1)
end
%    \end{macrocode}
%
%    \begin{macrocode}
base.marker = nodex.create_marker
return base
%    \end{macrocode}
% \iffalse
%</luamain>
% \fi
%
% \Finale
\endinput

\directlua{
  nodetree.set_option('engine', 'lualatex')
}
%    \end{macrocode}
%
%    \begin{macrocode}
\RequirePackage{kvoptions}
%    \end{macrocode}
%
%    \begin{macrocode}
\SetupKeyvalOptions{
  family=NT,
  prefix=NTK@
}
%    \end{macrocode}
%
%    \begin{macrocode}
\DeclareStringOption[term]{channel}
\define@key{NT}{channel}[]{\nodetreeoption[channel]{#1}}
%    \end{macrocode}
%
%    \begin{macrocode}
\DeclareStringOption[postlinebreak]{callback}
\define@key{NT}{callback}[]{\nodetreeoption[callback]{#1}}
%    \end{macrocode}
%
%    \begin{macrocode}
\DeclareStringOption[1]{verbosity}
\define@key{NT}{verbosity}[]{\nodetreeoption[verbosity]{#1}}
%    \end{macrocode}
%
%    \begin{macrocode}
\DeclareStringOption[colored]{color}
\define@key{NT}{color}[]{\nodetreeoption[color]{#1}}
%    \end{macrocode}
%
%    \begin{macrocode}
\DeclareStringOption[1]{unit}
\define@key{NT}{unit}[]{\nodetreeoption[unit]{#1}}
%    \end{macrocode}
%
%    \begin{macrocode}
\DeclareStringOption[1]{decimalplaces}
\define@key{NT}{decimalplaces}[]{\nodetreeoption[decimalplaces]{#1}}
%    \end{macrocode}
%
%    \begin{macrocode}
\DeclareStringOption[smyck]{theme}
%    \end{macrocode}
%
%    \begin{macrocode}
\DeclareStringOption[dark]{thememode}
%    \end{macrocode}
%
%    \begin{macrocode}
\DeclareStringOption[Ubuntu Mono]{font}
%    \end{macrocode}
%
%    \begin{macrocode}
\DeclareStringOption[\footnotesize]{fontsize}
%    \end{macrocode}
%

% Never load “heavy” packages like |mdframed| in default debug mode.
% They are way to slow.
%    \begin{macrocode}
\newif\ifdocumentationmode%
\documentationmodefalse%
\DeclareVoidOption{documentationmode}{%
  \RequirePackage{xcolor,mdframed,expl3,xparse}%
  \nodetreeoption[callback]{}%
  \documentationmodetrue%
}
%    \end{macrocode}
%
%    \begin{macrocode}
\ProcessKeyvalOptions{NT}
\directlua{
  nodetree.register_callbacks()
}
%    \end{macrocode}
%
% \begin{macro}{\nodetreeset}
%    \begin{macrocode}
\newcommand{\nodetreeset}[1]{%
  \setkeys{NT}{#1}%
}
%    \end{macrocode}
% \end{macro}
% Begin of the documentation mode.
%    \begin{macrocode}
\ifdocumentationmode
%    \end{macrocode}
%
% \begin{macro}{\NT@colors}
%    \begin{macrocode}
\ExplSyntaxOn
\def\NT@colors{
  \str_case_e:nn{\NTK@theme}{
    {bwdark}{
      \definecolor{NTblack}{gray}{0}
      \definecolor{NTred}{gray}{1}
      \definecolor{NTgreen}{gray}{1}
      \definecolor{NTyellow}{gray}{1}
      \definecolor{NTblue}{gray}{1}
      \definecolor{NTmagenta}{gray}{1}
      \definecolor{NTcyan}{gray}{1}
      \definecolor{NTgray}{gray}{1}
      \definecolor{NTblackbright}{gray}{0}
      \definecolor{NTredbright}{gray}{1}
      \definecolor{NTgreenbright}{gray}{1}
      \definecolor{NTyellowbright}{gray}{1}
      \definecolor{NTbluebright}{gray}{1}
      \definecolor{NTmagentabright}{gray}{1}
      \definecolor{NTcyanbright}{gray}{1}
      \definecolor{NTgraybright}{gray}{1}
    }
    {bwlight}{
      \definecolor{NTblack}{gray}{0}
      \definecolor{NTred}{gray}{0}
      \definecolor{NTgreen}{gray}{0}
      \definecolor{NTyellow}{gray}{0}
      \definecolor{NTblue}{gray}{0}
      \definecolor{NTmagenta}{gray}{0}
      \definecolor{NTcyan}{gray}{0}
      \definecolor{NTgray}{gray}{1}
      \definecolor{NTblackbright}{gray}{0}
      \definecolor{NTredbright}{gray}{0}
      \definecolor{NTgreenbright}{gray}{0}
      \definecolor{NTyellowbright}{gray}{0}
      \definecolor{NTbluebright}{gray}{0}
      \definecolor{NTmagentabright}{gray}{0}
      \definecolor{NTcyanbright}{gray}{0}
      \definecolor{NTgraybright}{gray}{1}
    }
    {terminalapp}{
      \definecolor{NTblack}{RGB}{0,0,0}
      \definecolor{NTred}{RGB}{194,54,33}
      \definecolor{NTgreen}{RGB}{37,188,36}
      \definecolor{NTyellow}{RGB}{173,173,39}
      \definecolor{NTblue}{RGB}{73,46,225}
      \definecolor{NTmagenta}{RGB}{211,56,211}
      \definecolor{NTcyan}{RGB}{51,187,200}
      \definecolor{NTgray}{RGB}{203,204,205}
      \definecolor{NTblackbright}{RGB}{129,131,131}
      \definecolor{NTredbright}{RGB}{252,57,31}
      \definecolor{NTgreenbright}{RGB}{49,231,34}
      \definecolor{NTyellowbright}{RGB}{234,236,35}
      \definecolor{NTbluebright}{RGB}{88,51,255}
      \definecolor{NTmagentabright}{RGB}{249,53,248}
      \definecolor{NTcyanbright}{RGB}{20,240,240}
      \definecolor{NTgraybright}{RGB}{233,235,235}
    }
    {xterm}{
      \definecolor{NTblack}{RGB}{0,0,0}
      \definecolor{NTred}{RGB}{205,0,0}
      \definecolor{NTgreen}{RGB}{0,205,0}
      \definecolor{NTyellow}{RGB}{205,205,0}
      \definecolor{NTblue}{RGB}{0,0,238}
      \definecolor{NTmagenta}{RGB}{205,0,205}
      \definecolor{NTcyan}{RGB}{0,205,205}
      \definecolor{NTgray}{RGB}{229,229,229}
      \definecolor{NTblackbright}{RGB}{127,127,127}
      \definecolor{NTredbright}{RGB}{255,0,0}
      \definecolor{NTgreenbright}{RGB}{0,255,0}
      \definecolor{NTyellowbright}{RGB}{255,255,0}
      \definecolor{NTbluebright}{RGB}{92,92,255}
      \definecolor{NTmagentabright}{RGB}{255,0,255}
      \definecolor{NTcyanbright}{RGB}{0,255,255}
      \definecolor{NTgraybright}{RGB}{255,255,255}
    }
    {smyck}{
      \definecolor{NTblack}{HTML}{212121}
      \definecolor{NTred}{HTML}{C75646}
      \definecolor{NTgreen}{HTML}{8EB33B}
      \definecolor{NTyellow}{HTML}{D0B03C}
      \definecolor{NTblue}{HTML}{72B3CC}
      \definecolor{NTmagenta}{HTML}{C8A0D1}
      \definecolor{NTcyan}{HTML}{218693}
      \definecolor{NTgray}{HTML}{B0B0B0}
      \definecolor{NTblackbright}{HTML}{5D5D5D}
      \definecolor{NTredbright}{HTML}{E09690}
      \definecolor{NTgreenbright}{HTML}{CDEE69}
      \definecolor{NTyellowbright}{HTML}{FFE377}
      \definecolor{NTbluebright}{HTML}{9CD9F0}
      \definecolor{NTmagentabright}{HTML}{FBB1F9}
      \definecolor{NTcyanbright}{HTML}{77DFD8}
      \definecolor{NTgraybright}{HTML}{F7F7F7}
    }
    {molokai}{
      \definecolor{NTblack}{HTML}{212121}
      \definecolor{NTred}{HTML}{fa2573}
      \definecolor{NTgreen}{HTML}{98e123}
      \definecolor{NTyellow}{HTML}{dfd460}
      \definecolor{NTblue}{HTML}{1080d0}
      \definecolor{NTmagenta}{HTML}{8700ff}
      \definecolor{NTcyan}{HTML}{43a8d0}
      \definecolor{NTgray}{HTML}{bbbbbb}
      \definecolor{NTblackbright}{HTML}{555555}
      \definecolor{NTredbright}{HTML}{f6669d}
      \definecolor{NTgreenbright}{HTML}{b1e05f}
      \definecolor{NTyellowbright}{HTML}{fff26d}
      \definecolor{NTbluebright}{HTML}{00afff}
      \definecolor{NTmagentabright}{HTML}{af87ff}
      \definecolor{NTcyanbright}{HTML}{51ceff}
      \definecolor{NTgraybright}{HTML}{ffffff}
    }
    {monokaisoda}{
      \definecolor{NTblack}{HTML}{1a1a1a}
      \definecolor{NTred}{HTML}{f4005f}
      \definecolor{NTgreen}{HTML}{98e024}
      \definecolor{NTyellow}{HTML}{fa8419}
      \definecolor{NTblue}{HTML}{9d65ff}
      \definecolor{NTmagenta}{HTML}{f4005f}
      \definecolor{NTcyan}{HTML}{58d1eb}
      \definecolor{NTgray}{HTML}{c4c5b5}
      \definecolor{NTblackbright}{HTML}{625e4c}
      \definecolor{NTredbright}{HTML}{f4005f}
      \definecolor{NTgreenbright}{HTML}{98e024}
      \definecolor{NTyellowbright}{HTML}{e0d561}
      \definecolor{NTbluebright}{HTML}{9d65ff}
      \definecolor{NTmagentabright}{HTML}{f4005f}
      \definecolor{NTcyanbright}{HTML}{58d1eb}
      \definecolor{NTgraybright}{HTML}{f6f6ef}
    }
  }
  \str_case_e:nn{\NTK@thememode}{
    {dark}{
      \definecolor{NTbackground}{named}{NTblack}
      \definecolor{NTfont}{named}{NTgraybright}
    }
    {light}{
      \definecolor{NTbackground}{named}{NTgraybright}
      \definecolor{NTfont}{named}{NTblack}
    }
  }
}
\ExplSyntaxOff
%    \end{macrocode}
% \end{macro}
%
% \begin{macro}{\NT@fonts}
%    \begin{macrocode}
\def\NT@fonts{
  \bfseries%
  \NTK@fontsize%
  \setmonofont{\NTK@font}%
  \ttfamily%
  \setlength{\parindent}{0pt}%
  \setlength{\parskip}{-0.9pt}%
}
%    \end{macrocode}
% \end{macro}
%
% \begin{macro}{\nodetreereset}
%    \begin{macrocode}
\def\nodetreereset{
  \nodetreeset{theme=monokaisoda,thememode=dark,font={Ubuntu Mono},fontsize=\tiny}
}
%    \end{macrocode}
% \end{macro}
%
% \begin{environment}{nodetreeexample}
%    \begin{macrocode}
\newenvironment{nodetreeexample}[1][]{
  \setkeys{NT}{#1}
  \NT@colors
  \begin{mdframed}[
    linecolor=black,
    backgroundcolor=NTbackground,
    fontcolor=NTfont,
  ]%
  \NT@fonts
}{
  \end{mdframed}%
}
%    \end{macrocode}
% \end{environment}
%
% \begin{macro}{\nodetreetermemulator}
%    \begin{macrocode}
\newcommand{\nodetreeterminalemulator}[2][]{
  \setkeys{NT}{#1}
  \begin{nodetreeexample}
  \input{#2.nttex}
  \end{nodetreeexample}
}
%    \end{macrocode}
% \end{macro}
%
%    \begin{macrocode}
\NewDocumentEnvironment { nodetreeShow } { +b } {
  \directlua{
    nodetree.compile_include('\luaescapestring{\unexpanded{#1}}')
  }
}{}
%    \end{macrocode}
%
% End of the documentation mode.
%    \begin{macrocode}
\fi
%    \end{macrocode}
%
% \iffalse
%</package>
% \fi
%
% \Finale
\endinput

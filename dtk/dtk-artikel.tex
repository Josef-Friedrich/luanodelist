%!TEX program = lualatex
\documentclass[ngerman]{dtk}
\ifluatex\else
  \usepackage[utf8]{inputenc}
\fi

\let\File\texttt
\let\Package\texttt

\begin{document}
\title{Das \texttt{nodetree}-Paket, Version 1.0}
\Author{Josef}{Friedrich}
    {Hirtengasse~5\\
     90443 Nürnberg\\
     \Email{josef@friedrich.rocks}}
\maketitle

\section{Über das Paket}

Mit dem \Package{nodetree} lässt sich DIE zentrale interne Datenstruktur
von Lua\TeX{} --- die \texttt{nodes} --- visualisieren. Das Paket wählt
dafür eine Baumansicht, wie sie auch vom den UNIX-Kommandozeilen-Befehl
\texttt{tree} verwendet wird.

\section{Über \texttt{nodes}}

Die Verarbeitung einer Quelltextdatei Lua\TeX{} in ein druckreifes PDF-%
Dokument erfolgt --- stark vereinfacht --- in drei Schritten:

\begin{enumerate}
\item Lua\TeX{} ersetzt alle Makros in nicht mehr weiter zu
expandierende \texttt{token}.

\item Lua\TeX{} berechnet aus diesen \texttt{token} die genaue Anordung
der grafischen bzw. typografischen Elemente im Dokument. Zur Speicherung
der Berechungen in unzähligen Zwischenschritten nutzt Lua\TeX{} als
Datenformat \texttt{nodes}. Am Ende dieser vielen Berechnungen ist ein
riesiger \texttt{node}-Baum entstanden. Die einzelnen \texttt{nodes}
wurden zu Zweigen und Ästen verkettet und diese wiederum zu einem
baumähnlichen Gebilde.

\item Ein Treiber generiert aus diesem \texttt{node}-Baum das gewünschte
Dokumentenformat, beispielsweise eine PDF-Datei.
\end{enumerate}

\section{Alternativen zu nodetree}

Das Script viznodelist.lua

https://gist.github.com/pgundlach/556247
\end{document}


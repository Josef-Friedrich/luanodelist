%!TEX program = luatex
%%
%% This is file `nodetree.tex',
%% generated with the docstrip utility.
%%
%% The original source files were:
%%
%% nodetree.dtx  (with options: `tex')
%% 
%% This is a generated file.
%% 
%% Copyright (C) 2015 by Josef Friedrich <josef@friedrich.rocks>
%% ----------------------------------------------------------------------
%% This work may be distributed and/or modified under the conditions of
%% the LaTeX Project Public License, either version 1.3c of this license
%% or (at your option) any later version. The latest version of this
%% license is in:
%% 
%%   http://www.latex-project.org/lppl.txt
%% 
%% and version 1.3c or later is part of all distributions of LaTeX
%% version 2008/05/05 or later.
%% 
\directlua{
  nodetree = require('nodetree')
  nodetree.set_option('engine', 'luatex')
}
\def\nodetreeoption[#1]#2{
  \directlua{
    nodetree.set_option('#1', '#2')
  }
}
\endinput
%%
%% End of file `nodetree.tex'.

\nodetreeregister{hpackquality}

\font\Large=cmb10 at 30pt
\Large
Archytas von Tarent (griechisch Archýtas; * wohl zwischen 435
und 410 v. Chr.; † wohl zwischen 355 und 350 v. Chr.) war ein antiker
griechischer Philosoph, Mathematiker, Musiktheoretiker, Physiker,
Ingenieur, Staatsmann und Feldherr.

Archytas wirkte in seiner Heimatstadt, der griechischen Kolonie Tarent
in Apulien. Als Philosoph gehörte er zur Richtung der Pythagoreer.
Bekannt ist er vor allem durch seine freundschaftliche Beziehung zu
Platon, durch die angeblich von ihm erfundene fliegende Taube und durch
ein Gedankenexperiment, mit dem er die Unendlichkeit des Universums
beweisen wollte. Von seinen Schriften, die insbesondere Themen der
Mathematik und der Musik behandelten, sind nur wenige Fragmente erhalten
geblieben.

Eine maßgebliche Rolle spielte Archytas politisch und militärisch als
leitender Staatsmann und Stratege seiner Heimatstadt und eines von ihr
geführten Bundes griechischer Kolonien Süditaliens. Seine militärischen
Erfolge verschafften ihm hohe Autorität. Innenpolitisch setzte er sich
für sozialen Ausgleich ein, wobei er es für möglich hielt, ein
Gerechtigkeitskonzept wissenschaftlich zu begründen und damit Konsens
herbeizuführen.
\bye
